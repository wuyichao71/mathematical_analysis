\chapter{不定积分}
% 陈老视频66(2023.01.29)
\section{不定积分的概念和运算法则}
\subsection{不定积分的概念}
在之前讨论人口模型和液体过滤问题时, 实际上我们已经涉及了积分的概念。
\begin{definition}
    在某个区间上, $F'(x) = f(x)$, 则称$F(x)$是$f(x)$的一个原函数。
\end{definition}
\begin{remark}
    这里说一个, 是因为原函数不唯一。设$G(x)$是该区间上的另一个原函数, 则$(F(x)-G(x))' = f(x) - f(x) = 0$, 则$F(x) - G(x) = C$, $C$为常数。
\end{remark}

\begin{definition}
    一个函数$f(x)$的原函数的全体称为这个函数的不定积分, 记作$\int f(x)\D x$。其中, $\int$为积分号, $f(x)$为被积函数, $x$是积分变量。不定积分可以记为:
    \begin{equation*}
        \int f(x) \D x = F(x) + C
    \end{equation*}
\end{definition}

\begin{example}
    求:
    \begin{equation*}
        \int sin(x) \D x
    \end{equation*}
\end{example}
\begin{solution}
    
\end{solution}

% 陈老视频66(2023.01.31)
\begin{example}
    求:
    \begin{equation*}
        \int x^\alpha \D x \quad (\alpha \neq -1)
    \end{equation*}
\end{example}
\begin{solution}
    
\end{solution}

\begin{example}
    求:
    \begin{equation*}
        \int \frac{1}{x} \D x
    \end{equation*}
\end{example}
\begin{solution}
    
\end{solution}

不定积分表

\subsection{不定积分的线性性质}
\begin{theorem}[不定积分的线性性质]
    设$f(x)$, $g(x)$的原函数都存在, $k_1, k_2$是任意常数, 则:
    \begin{equation*}
        \int [k_1 f(x) + k_2 g(x)] \D x = k_1 \int f(x) \D x + k_2 \int g(x) \D x
    \end{equation*}
\end{theorem}
\begin{proof}
    
\end{proof}

\begin{example}
    求:
    \begin{equation*}
        \int \tan^2(x) \D x
    \end{equation*}
\end{example}
\begin{solution}
    
\end{solution}

\begin{example}
    求:
    \begin{equation*}
        \int sin^2(\frac{x}{2}) \D x
    \end{equation*}
\end{example}
\begin{solution}
    
\end{solution}

\begin{example}
    求:
    \begin{equation*}
        \int \frac{\left(x+\sqrt{x}\right)\left(x-2\sqrt{x}\right)^2}{\sqrt{x}} \D x
    \end{equation*}
\end{example}
\begin{solution}
    
\end{solution}

\begin{example}
    求:
    \begin{equation*}
        \int \frac{x^4}{1+x^2} \D x
    \end{equation*}
\end{example}

\begin{example}
    曲线$g=f(x)$在没一点$(x, f(x))$切线的斜率为$x^2$, 且曲线经过$(3, 2)$, 求$y = f(x)$。
\end{example}
\begin{solution}
    
\end{solution}

% 陈老视频67(2023.02.01)
\section{换元积分法和分步积分法}
\subsection{第一类换元积分法}
\begin{theorem}[第一类换元积分法]
    若要求$\int f(x) dx$, 若$f(x)$可以写成$f(x)=\widetilde{f}(g(x))g'(x)$, 且$\int \widetilde{f}(u) \D u = F(u) + C$, 则$\int f(x) \D x = F(g(x)) + C$。    
\end{theorem}
\begin{proof}
    
\end{proof}
\begin{remark}
    第一类换元积分法又称为``凑微分法''。
\end{remark}

\begin{example}
    求:
    \begin{equation*}
        \int \frac{\D x}{x - a}
    \end{equation*}
\end{example}
\begin{solution}
    
\end{solution}

\begin{example}
    求:
    \begin{equation*}
        \int \frac{\D x}{(x-a)^n}
    \end{equation*}
\end{example}
\begin{solution}
    
\end{solution}

\begin{example}
    求:
    \begin{equation*}
        \int \frac{\D x}{x^2 - a^2}
    \end{equation*}
\end{example}
\begin{solution}
    
\end{solution}

\begin{example}
    求:
    \begin{equation*}
        \int \frac{\D x}{x^2 + a^2}
    \end{equation*}
\end{example}
\begin{solution}
    
\end{solution}

\begin{example}
    求:
    \begin{equation*}
        \int \frac{\D x}{\sqrt{a^2-x^2}}
    \end{equation*}
\end{example}
\begin{solution}
    
\end{solution}

\begin{example}
    求:
    \begin{equation*}
        \int \tan(x) \D x
    \end{equation*}
\end{example}
\begin{solution}
    
\end{solution}

\begin{example}
    求:
    \begin{equation*}
        \int \cot(x) \D x
    \end{equation*}
\end{example}
\begin{solution}
    
\end{solution}

\begin{example}
    求:
    \begin{equation*}
        \int \sec(x) \D x
    \end{equation*}
\end{example}
\begin{solution}
    
\end{solution}

\begin{example}
    求:
    \begin{equation*}
        \int \csc(x) \D x
    \end{equation*}
\end{example}
\begin{solution}
    
\end{solution}

\begin{example}
    求:
    \begin{equation}
        \int \frac{\D x}{\sqrt{x}(1+x)}
    \end{equation}
\end{example}
\begin{solution}
    
\end{solution}

\begin{example}
    求:
    \begin{equation}
        \int \sin(mx)\cos(nx) dx \quad (m \neq n)
    \end{equation}
\end{example}
\begin{solution}
    
\end{solution}

\begin{example}
    求:
    \begin{equation}
        \int \sin(mx) \cos(nx) \D x
    \end{equation}
\end{example}
\begin{solution}
    
\end{solution}

\begin{example}
    求:
    \begin{equation}
        \int \cos(mx) \cos(nx) \D x
    \end{equation}
\end{example}
\begin{solution}
    
\end{solution}

\begin{example}
    求:
    \begin{equation}
        \int \sin(mx) \sin(nx) \D x
    \end{equation}
\end{example}
\begin{solution}
    
\end{solution}

% 陈老视频68(2023.02.03)
\subsection{第二类换元积分法}
\begin{theorem}[第二类换元积分法]
    若要求$\int f(x) dx$, 若存在$x = \phi(t)$使得$\int f(\phi(t))\phi'(t) \D t = F(t) + C$, 则$\int f(x) \D x = F(\phi^{-1}(x)) + C$
\end{theorem}
\begin{remark}
    \begin{equation*}
        \int f(x) \D x = \int f(\phi(t)) \D \phi(t) = \int f(\phi(t)) \phi'(t) \D t
    \end{equation*}
    若$\int f(\phi(t)) \phi'(t) \D t = F(t) + C$, 则$\int f(x) \D x = F(\phi^{-1}(t)) + C$。
\end{remark}
\begin{example}
    求:
    \begin{equation*}
        \int \sqrt{a^2-x^2} \D x
    \end{equation*}
\end{example}
\begin{solution}
    
\end{solution}

\begin{example}
    求:
    \begin{equation*}
        \int \frac{\D x}{\sqrt{x^2-a^2}}
    \end{equation*}
\end{example}
\begin{solution}
    
\end{solution}

\begin{example}
    求:
    \begin{equation}
        \int \frac{\D x}{\sqrt{x^2+a^2}}
    \end{equation}
\end{example}
\begin{solution}
    
\end{solution}

\begin{example}
    求:
    \begin{equation*}
        \int x(2x-1)^{100} \D x
    \end{equation*}
\end{example}
\begin{solution}
    
\end{solution}

\begin{example}
    求:
    \begin{equation*}
        \int \frac{\D x}{x^2\sqrt{1+x^2}}
    \end{equation*}
    请分别用第一类和第二类换元法(两种)求解。
\end{example}
\begin{solution}
    
\end{solution}

% 陈老视频69(2023.02.04)
\subsection{分步积分法}
\begin{theorem}[分步积分法]
    若要求$\int u(x)v'(x)\D x$, 且已知$\int v(x)u'(x) \D x = F(x) + c$, 则:
    \begin{equation*}
        \int u(x)v'(x) \D x = u(x)v(x) - F(x) + C
    \end{equation*}
\end{theorem}

\begin{example}
    求:
    \begin{equation*}
        \int x \cos(x) \D x
    \end{equation*}
\end{example}
\begin{solution}
    
\end{solution}

\begin{example}
    求:
    \begin{equation*}
        \int x^2\e^x \D x
    \end{equation*}
\end{example}
\begin{solution}
    
\end{solution}
\begin{remark}
    \begin{enumerate}
        \item 令$P_n(x)$是$n$次多项式:
        \begin{equation*}
            \int P_n(x)\sin(\alpha x) \D x, \quad \int P_n(x)\cos(\lambda x) \D x, \quad \int P_n(x) \e^{\lambda x} \D x
        \end{equation*}
        都可以用分步积分法, 其中要把$\sin(\alpha x), \cos(\lambda x) \e^{\lambda x}$放入微分符号内。
        \item 对于$\int P_n(x) \arcsin(x) \D x, \quad \int P_n(x) \arctan(x) \D x,, \quad \int P_n(x)\ln(x) \D x$, 需要把$P_n(x)$放入微分符号内。
    \end{enumerate}
\end{remark}

\begin{example}
    求:
    \begin{equation*}
        \int \ln(x) \D x
    \end{equation*}
\end{example}
\begin{solution}
    
\end{solution}

\begin{example}
    求:
    \begin{equation*}
        \int x\arctan(x) \D x
    \end{equation*}
\end{example}
\begin{solution}
    
\end{solution}

\begin{example}
    求:
    \begin{equation*}
        \int \frac{x}{1+cos(x)} \D x
    \end{equation*}
\end{example}
\begin{solution}
    
\end{solution}
考虑$\int \e^{\lambda x} \cos(\alpha x) \D x, \quad \int \sin(\lambda x) \D x, \quad \int \sqrt{a^2 + x^2} \D x$, 如果使用分步积分法会无法简化。
\begin{example}
    求:
    \begin{equation*}
        \int \e^x \sin(x) \D x
    \end{equation*}
\end{example}
\begin{solution}
    
\end{solution}

\begin{example}
    求:
    \begin{equation*}
        \int \sqrt{x^2+a^2} \D x
    \end{equation*}
\end{example}
\begin{solution}
    
\end{solution}

\begin{example}
    求:
    \begin{equation*}
        \int \sqrt{x^2-a^2} \D x
    \end{equation*}
\end{example}
\begin{solution}
    
\end{solution}
\begin{remark}
    现在我们已经求解了$\int \frac{\D x}{\sqrt{a^2-x^2}}, \int \frac{\D x}{\sqrt{x^2 \pm a^2}}, \int \sqrt{a^2-x^2} \D x, \int \sqrt{x^2\pm a^2} \D x$。
\end{remark}

% 陈老视频70(2023.02.05)
\begin{remark}
    要求求$\int f^n(x) \D x$, 可以通过分部积分法降低$f(x)$的次数, 得到一个递推公式。
\end{remark}

\begin{example}
    求:
    \begin{equation}
        \int \frac{\D x}{(x^2+a^2)^n}
    \end{equation}
\end{example}
\begin{solution}
    
\end{solution}

积分表

\begin{example}
    求:
    \begin{equation*}
        \int \sqrt{x^2-2x+5} \D x
    \end{equation*}
\end{example}
\begin{solution}
    
\end{solution}

\begin{example}
    求:
    \begin{equation*}
        \int (x+1)\sqrt{x^2=2x+5} \D x
    \end{equation*}
\end{example}
\begin{solution}
    
\end{solution}

% 陈老视频71(2023.02.06)
\section{有理函数的不定积分及其应用}
$\int \frac{\sin(x)}{x} \D x$, $\int \e^{\pm x^2} \D x$, $\int \sqrt{1-k^2\sin^2(x)} \D x \quad (0 < k^2 < 1)$无法用初等函数表示。

\begin{definition}[有理函数]
    记有理函数$R(x) = \frac{p_m(x)}{q_n(x)}$, $p_m(x)$, $q_n(x)$分别是次数为$m$, $n$的多项式。
\end{definition}

\begin{theorem}
    若$f(x)$是有理函数, 则$\int f(x) \D x$是初等函数。
\end{theorem}
\begin{proof}
    不妨设$p_m$, $q_n$没有公因式, $q_n(x)$最高次的系数为1, 且$m < n$, 即$R(x)$是真分数。

    \begin{equation*}
        q_n(x) = \prod_{k=1}^{i}(x-\alpha_k)^{m_k}\prod_{k=1}^{j}(x^2+2\xi_k x + \eta_k^2)^{n_k}
    \end{equation*}
\end{proof}

\begin{theorem}
    设有$\frac{p(x)}{q(x)}$, 多项式$q(x)$有$k$重根, 即$q(x)=(x-\alpha)^kq_1(x), q_1(\alpha) \neq 0$, 则存在实数$\lambda$和$p_1(x)$($p_1(x)$次数小于$(x-\alpha)^{k-1}q_1(x)$的次数), 使得
    \begin{equation}
        \frac{p(x)}{q(x)} = \frac{\lambda}{(x-\alpha)^k} + \frac{p_1(x)}{(x-\alpha)^{k-1}q_1(x)}
    \end{equation}
\end{theorem}
% 陈老视频72(2023.02.07)
\begin{proof}
    
\end{proof}

\begin{theorem}
    设有$\frac{p(x)}{q(x)}$, 多项式$q(x)$有$l$重共轭虚根$\beta\pm i\lambda$, 即$q(x) = (x^2+2\xi x+ \eta^2)^l q*(x), q*(\beta\pm i\lambda) \neq 0$, 其中$\xi = -\beta, \eta^2 = \gamma^2+\beta^2$, 则存在实数$\mu$, $\nu$, 多项式$p*(x)$($p*$次数小于$(x^2+2\xi x+\eta^2)^{l-1}q*(x)$的次数), 使得
    \begin{equation}
        \frac{p(x)}{q(x)} = \frac{\mu x + \nu}{(x^2+2\xi x+\eta^2)^l} + \frac{q*(x)}{(x^2+2\xi x+\eta^2)^{l-1}q*(x)}
    \end{equation}
\end{theorem}
\begin{proof}
    
\end{proof}

最后$\frac{p_m(x)}{q_n(x)}$可以写成
\begin{equation*}
    \frac{p_m(x)}{q_n(x)} = \sum_{k=1}^{i}\sum_{r=1}^{m_k}\frac{\lambda_r}{(x-\alpha_k)^r}+\sum_{k=1}^{j}\sum_{r=1}^{n_k}\frac{\mu_r x+ \nu_r}{(x^2+2\xi_k x +\eta_k^2)^r}
\end{equation*}
部分分式

\begin{equation*}
    \int \frac{\D x}{(x-\alpha)^n} = \left\{
        \begin{aligned}
            &\ln|x-\alpha| + C &\quad n = 1 \\
            &\frac{1}{-n+1}(x-\alpha)^{-n+1} + C &\quad n \ge 2
        \end{aligned}
    \right.    
\end{equation*}

\begin{equation*}
    \int \frac{\mu x + \nu}{(x^2+2\xi x+\eta^2)^n}\D x = \frac{\mu}{2}\int\frac{2x+2\xi}{(x^2+2\xi x+\eta^2)^n} \D x + (\nu-\mu \xi)\int\frac{\D x}{(x^2+2\xi x+\eta^2)^n}
\end{equation*}

\begin{equation*}
    \int\frac{(2x+2\xi)\D x}{(x^2+2\xi x+\eta^2)^n} = \left\{
        \begin{aligned}
            &\ln|x^2+2\xi x +\eta^2| + C & \quad n = 1 \\
            &\frac{1}{-n+1}(x^2+2\xi x + \eta^2)^{-n+1} + C & \quad n \ge 2
        \end{aligned}
    \right.
\end{equation*}

令
\begin{equation*}
    I_n = \int \frac{\D x}{(x^2+2\xi x + \eta^2)^n}
\end{equation*}

则
\begin{equation*}
    \begin{aligned}
        I_n &= \int \frac{\D (x+\xi)}{((x+\xi)^2+a^2)^n} = \int\frac{\D u}{(u^2+a^2)^n} \\
        &= \frac{1}{2(\eta^2-\xi^2)(n-1)}\left[(2n-3)I_{n-1}+\frac{x+\xi}{(x^2+2\xi x+\eta^2)^{n-1}}\right]
    \end{aligned}
\end{equation*}

\begin{equation*}
    I_1 = \int\frac{\D x}{x^2+2\xi x+\eta^2} = \frac{1}{\sqrt{\eta^2-\xi^2}}\arctan\left(\frac{x+\xi}{\sqrt{\eta^2-\xi^2}}\right) + C
\end{equation*}

\begin{example}
    求:
    \begin{equation*}
        \int \frac{4x^3-13x^2+3x+8}{(x+1)(x-2)(x-1)^2} \D x
    \end{equation*}
\end{example}
\begin{solution}
    
\end{solution}
\begin{remark}
    待定系数法
\end{remark}

\begin{example}
    求:
    \begin{equation*}
        \int \frac{x^4+x^3+3x^2-1}{(x^2+1)^2(x-1)} \D x
    \end{equation*}
\end{example}
\begin{solution}
    
\end{solution}

% 陈老视频73(2023.02.08)
可化为有理函数不定积分的例子

(1) 带有根式的不定积分

$R(u, v)$是关于$u$, $v$的有理函数, 即$R(u, v) = \frac{p(u, v)}{q(u, v)}$, $p(u, v)$, $q(u, v)$都是关于$u$, $v$的多项式。

现在令$u = x$, $v = \sqrt[n]{\frac{\xi x + u}{\mu x + v}}$, 通过令$t = \sqrt[n]{\frac{\xi x + u}{\mu x + v}}$可以化为有理函数。

\begin{example}
    求:
    \begin{equation*}
        \int \frac{x \D x}{\sqrt{4x-3}}
    \end{equation*}    
\end{example}
\begin{solution}
    
\end{solution}

\begin{example}
    求:
    \begin{equation*}
        \int \frac{\D x}{x(\sqrt[3]{x}-\sqrt{x})}
    \end{equation*}
\end{example}
\begin{solution}
    
\end{solution}

\begin{example}
    求:
    \begin{equation*}
        \int \frac{\sqrt{1+x}}{x\sqrt{1-x}} \D x
    \end{equation*}
\end{example}
\begin{solution}
    
\end{solution}

(1)考虑$\sqrt[n]{(\xi x+\eta)^i(\mu x + \nu)^j}$, 其中$i + j = kn, k \in \mathbb{N}^+$, 则该函数的积分可以化为有理函数。因为
\begin{equation*}
    \sqrt[n]{(\xi x+\eta)^i(\mu x + \nu)^j} = \sqrt[n]{(\xi x+\eta)^{kn}\frac{(\mu x + \nu)^j}{(\xi x+\eta)^j}} = (\xi x+\eta)^k\sqrt[n]{\left(\frac{\mu x + \nu}{\xi x+\eta}\right)^j}
\end{equation*}

% 陈老视频74(2023.02.09)
\begin{example}
    求:
    \begin{equation*}
        \int \frac{\D x}{\sqrt[3]{(x-1)^2(x+1)^4}}
    \end{equation*}
\end{example}
\begin{solution}
    
\end{solution}

$R(u, v)$定义如(1), 则$R(\sin(x), \cos(x))$代表了所有三角函数的有理函数。
考虑
\begin{equation*}
    I = \int R(\sin(x), \cos(x)) \D x
\end{equation*}

令$\tan\left(\frac{x}{2}\right) = t$, 则:
% \begin{equation*}
\begin{gather*}
    \sin(x) = \frac{2t}{1+t^2} \\
    \cos(x) = \frac{1-t^2}{1+t^2} \\ 
    I = \int R\left(\frac{2t}{1+t^2}, \frac{1-t^2}{1+t^2}\right)\frac{2}{1+t^2}\D t
\end{gather*}
% \end{equation*}

\begin{example}
    求:
    \begin{equation*}
        \int \frac{\D x}{4 + 4\sin(x)+\cos(x)}
    \end{equation*}
\end{example}

\begin{example}
    求:
    \begin{equation*}
        \int \frac{\cot(x) \D x}{1+ \sin(x)} 
    \end{equation*}
\end{example}
\begin{solution}
    \framebox{解法一}:

    \framebox{解法二}:
\end{solution}

