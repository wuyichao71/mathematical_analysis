\chapter{不定积分}
% 陈老视频66(2023.01.29)
\section{不定积分的概念和运算法则}
\subsection{不定积分的概念}
在之前讨论人口模型和液体过滤问题时, 实际上我们已经涉及了积分的概念。
\begin{definition}
    在某个区间上, $F'(x) = f(x)$, 则称$F(x)$是$f(x)$的一个原函数。
\end{definition}
\begin{remark}
    这里说一个, 是因为原函数不唯一。设$G(x)$是该区间上的另一个原函数, 则$(F(x)-G(x))' = f(x) - f(x) = 0$, 则$F(x) - G(x) = C$, $C$为常数。
\end{remark}

\begin{definition}
    一个函数$f(x)$的原函数的全体称为这个函数的不定积分, 记作$\int f(x)\D x$。其中, $\int$为积分号, $f(x)$为被积函数, $x$是积分变量。不定积分可以记为:
    \begin{equation*}
        \int f(x) \D x = F(x) + C
    \end{equation*}
\end{definition}

\begin{example}
    求:
    \begin{equation*}
        \int sin(x) \D x
    \end{equation*}
\end{example}
\begin{solution}
    
\end{solution}

% 陈老视频66(2023.01.31)
\begin{example}
    求:
    \begin{equation*}
        \int x^\alpha \D x \quad (\alpha \neq -1)
    \end{equation*}
\end{example}
\begin{solution}
    
\end{solution}

\begin{example}
    求:
    \begin{equation*}
        \int \frac{1}{x} \D x
    \end{equation*}
\end{example}
\begin{solution}
    
\end{solution}

不定积分表

\subsection{不定积分的线性性质}
\begin{theorem}[不定积分的线性性质]
    设$f(x)$, $g(x)$的原函数都存在, $k_1, k_2$是任意常数, 则:
    \begin{equation*}
        \int [k_1 f(x) + k_2 g(x)] \D x = k_1 \int f(x) \D x + k_2 \int g(x) \D x
    \end{equation*}
\end{theorem}
\begin{proof}
    
\end{proof}

\begin{example}
    求:
    \begin{equation*}
        \int \tan^2(x) \D x
    \end{equation*}
\end{example}
\begin{solution}
    
\end{solution}

\begin{example}
    求:
    \begin{equation*}
        \int sin^2(\frac{x}{2}) \D x
    \end{equation*}
\end{example}
\begin{solution}
    
\end{solution}

\begin{example}
    求:
    \begin{equation*}
        \int \frac{\left(x+\sqrt{x}\right)\left(x-2\sqrt{x}\right)^2}{\sqrt{x}} \D x
    \end{equation*}
\end{example}
\begin{solution}
    
\end{solution}

\begin{example}
    求:
    \begin{equation*}
        \int \frac{x^4}{1+x^2} \D x
    \end{equation*}
\end{example}

\begin{example}
    曲线$g=f(x)$在没一点$(x, f(x))$切线的斜率为$x^2$, 且曲线经过$(3, 2)$, 求$y = f(x)$。
\end{example}
\begin{solution}
    
\end{solution}

% 陈老视频67(2023.02.01)
\section{换元积分法和分步积分法}
\subsection{第一类换元积分法}
\begin{theorem}[第一类换元积分法]
    若要求$\int f(x) dx$, 若$f(x)$可以写成$f(x)=\widetilde{f}(g(x))g'(x)$, 且$\int \widetilde{f}(u) \D u = F(u) + C$, 则$\int f(x) \D x = F(g(x)) + C$。    
\end{theorem}
\begin{proof}
    
\end{proof}
\begin{remark}
    第一类换元积分法又称为``凑微分法''。
\end{remark}

\begin{example}
    求:
    \begin{equation*}
        \int \frac{\D x}{x - a}
    \end{equation*}
\end{example}
\begin{solution}
    
\end{solution}

\begin{example}
    求:
    \begin{equation*}
        \int \frac{\D x}{(x-a)^n}
    \end{equation*}
\end{example}
\begin{solution}
    
\end{solution}

\begin{example}
    求:
    \begin{equation*}
        \int \frac{\D x}{x^2 - a^2}
    \end{equation*}
\end{example}
\begin{solution}
    
\end{solution}

\begin{example}
    求:
    \begin{equation*}
        \int \frac{\D x}{x^2 + a^2}
    \end{equation*}
\end{example}
\begin{solution}
    
\end{solution}

\begin{example}
    求:
    \begin{equation*}
        \int \frac{\D x}{\sqrt{a^2-x^2}}
    \end{equation*}
\end{example}
\begin{solution}
    
\end{solution}

\begin{example}
    求:
    \begin{equation*}
        \int \tan(x) \D x
    \end{equation*}
\end{example}
\begin{solution}
    
\end{solution}

\begin{example}
    求:
    \begin{equation*}
        \int \cot(x) \D x
    \end{equation*}
\end{example}
\begin{solution}
    
\end{solution}

\begin{example}
    求:
    \begin{equation*}
        \int \sec(x) \D x
    \end{equation*}
\end{example}
\begin{solution}
    
\end{solution}

\begin{example}
    求:
    \begin{equation*}
        \int \csc(x) \D x
    \end{equation*}
\end{example}
\begin{solution}
    
\end{solution}

\begin{example}
    求:
    \begin{equation}
        \int \frac{\D x}{\sqrt{x}(1+x)}
    \end{equation}
\end{example}
\begin{solution}
    
\end{solution}

\begin{example}
    求:
    \begin{equation}
        \int \sin(mx)\cos(nx) dx \quad (m \neq n)
    \end{equation}
\end{example}
\begin{solution}
    
\end{solution}

\begin{example}
    求:
    \begin{equation}
        \int \sin(mx) \cos(nx) \D x
    \end{equation}
\end{example}
\begin{solution}
    
\end{solution}

\begin{example}
    求:
    \begin{equation}
        \int \cos(mx) \cos(nx) \D x
    \end{equation}
\end{example}
\begin{solution}
    
\end{solution}

\begin{example}
    求:
    \begin{equation}
        \int \sin(mx) \sin(nx) \D x
    \end{equation}
\end{example}
\begin{solution}
    
\end{solution}

% 陈老视频68(2023.02.03)
\subsection{第二类换元积分法}
\begin{theorem}[第二类换元积分法]
    若要求$\int f(x) dx$, 若存在$x = \phi(t)$使得$\int f(\phi(t))\phi'(t) \D t = F(t) + C$, 则$\int f(x) \D x = F(\phi^{-1}(x)) + C$
\end{theorem}
\begin{remark}
    \begin{equation*}
        \int f(x) \D x = \int f(\phi(t)) \D \phi(t) = \int f(\phi(t)) \phi'(t) \D t
    \end{equation*}
    若$\int f(\phi(t)) \phi'(t) \D t = F(t) + C$, 则$\int f(x) \D x = F(\phi^{-1}(t)) + C$。
\end{remark}
\begin{example}
    求:
    \begin{equation*}
        \int \sqrt{a^2-x^2} \D x
    \end{equation*}
\end{example}
\begin{solution}
    
\end{solution}

\begin{example}
    求:
    \begin{equation*}
        \int \frac{\D x}{\sqrt{x^2-a^2}}
    \end{equation*}
\end{example}
\begin{solution}
    
\end{solution}

\begin{example}
    求:
    \begin{equation}
        \int \frac{\D x}{\sqrt{x^2+a^2}}
    \end{equation}
\end{example}
\begin{solution}
    
\end{solution}

\begin{example}
    求:
    \begin{equation*}
        \int x(2x-1)^{100} \D x
    \end{equation*}
\end{example}
\begin{solution}
    
\end{solution}

\begin{example}
    求:
    \begin{equation*}
        \int \frac{\D x}{x^2\sqrt{1+x^2}}
    \end{equation*}
    请分别用第一类和第二类换元法(两种)求解。
\end{example}
\begin{solution}
    
\end{solution}

% 陈老视频69(2023.02.04)
\subsection{分步积分法}
\begin{theorem}[分步积分法]
    若要求$\int u(x)v'(x)\D x$, 且已知$\int v(x)u'(x) \D x = F(x) + c$, 则:
    \begin{equation*}
        \int u(x)v'(x) \D x = u(x)v(x) - F(x) + C
    \end{equation*}
\end{theorem}

\begin{example}
    求:
    \begin{equation*}
        \int x \cos(x) \D x
    \end{equation*}
\end{example}
\begin{solution}
    
\end{solution}

\begin{example}
    求:
    \begin{equation*}
        \int x^2\e^x \D x
    \end{equation*}
\end{example}
\begin{solution}
    
\end{solution}
\begin{remark}
    \begin{enumerate}
        \item 令$P_n(x)$是$n$次多项式:
        \begin{equation*}
            \int P_n(x)\sin(\alpha x) \D x, \quad \int P_n(x)\cos(\lambda x) \D x, \quad \int P_n(x) \e^{\lambda x} \D x
        \end{equation*}
        都可以用分步积分法, 其中要把$\sin(\alpha x), \cos(\lambda x) \e^{\lambda x}$放入微分符号内。
        \item 对于$\int P_n(x) \arcsin(x) \D x, \quad \int P_n(x) \arctan(x) \D x,, \quad \int P_n(x)\ln(x) \D x$, 需要把$P_n(x)$放入微分符号内。
    \end{enumerate}
\end{remark}

\begin{example}
    求:
    \begin{equation*}
        \int \ln(x) \D x
    \end{equation*}
\end{example}
\begin{solution}
    
\end{solution}

\begin{example}
    求:
    \begin{equation*}
        \int x\arctan(x) \D x
    \end{equation*}
\end{example}
\begin{solution}
    
\end{solution}

\begin{example}
    求:
    \begin{equation*}
        \int \frac{x}{1+cos(x)} \D x
    \end{equation*}
\end{example}
\begin{solution}
    
\end{solution}
考虑$\int \e^{\lambda x} \cos(\alpha x) \D x, \quad \int \sin(\lambda x) \D x, \quad \int \sqrt{a^2 + x^2} \D x$, 如果使用分步积分法会无法简化。
\begin{example}
    求:
    \begin{equation*}
        \int \e^x \sin(x) \D x
    \end{equation*}
\end{example}
\begin{solution}
    
\end{solution}

\begin{example}
    求:
    \begin{equation*}
        \int \sqrt{x^2+a^2} \D x
    \end{equation*}
\end{example}
\begin{solution}
    
\end{solution}

\begin{example}
    求:
    \begin{equation*}
        \int \sqrt{x^2-a^2} \D x
    \end{equation*}
\end{example}
\begin{solution}
    
\end{solution}
\begin{remark}
    现在我们已经求解了$\int \frac{\D x}{\sqrt{a^2-x^2}}, \int \frac{\D x}{\sqrt{x^2 \pm a^2}}, \int \sqrt{a^2-x^2} \D x, \int \sqrt{x^2\pm a^2} \D x$。
\end{remark}