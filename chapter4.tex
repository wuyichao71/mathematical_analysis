% 陈老视频39
\chapter{微分}
\section{微分和导数}
\subsection{微分}
考虑$y = f(x)$, 当$x \to x + \Delta x$时, $f(x) \to f(x+\Delta x)$, 令$\Delta y = f(x+\Delta x) - f(x)$。应该怎么简单地表示$\Delta y$?
\begin{definition}[微分的定义]
    $x_0 \in D_f$, 若存在只与$x_0$有关, 与$\Delta x$无关的$g(x_0)$,使得当$\Delta x \to 0$时:
    \[ \Delta y = g(x_0)\Delta x + o(\Delta x) \]
    则称$f(x)$在$x_0$可微。

    若$f(x)$在区间X的每一点可微, 则称$f(x)$在区间X可微。

    $g(x_0)\Delta x$称为$\Delta y$的线性主要部分。

    $\Delta x \to 0$, 记$\Delta x$为$dx$, 若$f(x)$在$x$点可微, 则有$\Delta y = g(x)\Delta x + o(\Delta x), (\Delta x \to 0)$。则记$\Delta y$为$dy$, 并将上式写为$dy = g(x)dx $。
\end{definition}

\begin{proposition}
    \[ y = f(x) = x^2 \]
    对$\forall x \in (-\infty, +\infty)$, 求微分表示。
\end{proposition}
\begin{proof}
    
\end{proof}

\begin{proposition}
    \[ y = f(x) = \sqrt[3]{x^2} \]
    考虑$f$在$x_0 = 0$是否可微。
\end{proposition}
\begin{proof}
    
\end{proof}

可微$\Rightarrow$连续
\subsection{导数}
$y = f(x)$在$x_0$可微, 则$\Delta y = g(x_0)\Delta x + o(\Delta x), (\Delta x \to 0)$, 那么:
\[ \frac{\Delta y}{\Delta x} = g(x_0) + \frac{o(\Delta x)}{\Delta x} \]
令$\Delta x \to 0$, 则
\[ \funclim{\Delta x}{0}{\frac{\Delta y}{\Delta x}} = g(x_0)\]
\begin{definition}
    设$x_0 \in D_f$, 若极限
    \[ \funclim{\Delta x}{0}{\frac{\Delta y}{\Delta x}} = \funclim{\Delta x}{0}{\frac{f(x+\Delta x) - f(x)}{\Delta x}} \]
    存在, 则称$f(x)$在$x_0$可导, 记这个极限值为$f'(x_0)$(或$y'(x_0)$, $\left.\frac{dy}{dx}\right|_{x = x_0}$, $\left.\frac{df}{dx}\right|_{x = x_0}$)。
\end{definition}
$f(x)$可导的范围是$D_f$的子集, 于是我们可以得到在这子集上的$f(x)$的导函数, 记为$f'(x)$(或$y'(x)$, $\frac{dy}{dx}$, $\frac{df}{dx}$)。

可微$\Rightarrow$可导, 且$f'(x_0) = g(x_0)$。

可导是否一定可微?

可导, 则:
\[ \funclim{\Delta x}{0}{\frac{\Delta y}{\Delta x}} = f'(x_0)\]
则:
\[  \funclim{\Delta x}{0}{\left(\frac{\Delta y}{\Delta x} - f'(x_0) \right)} = 0 \]
\[ \frac{\Delta y}{\Delta x} - f'(x_0) = o(1), (\Delta x \to 0) \]
\[ \Delta y = f'(x_0)\Delta x + o(1)\Delta x \]
\[ \Delta y = f'(x_0)\Delta x + o(\Delta x) \]
即, 可导$\Rightarrow$可微(一元函数下)。

% 陈老视频40
\section{导数的意义与性质}
\begin{proposition}
    抛物线:
    \[ y^2 = 2px \]
    $(x_0, y_0)$是抛物线上一点, 求过$(x_0, y_0)$的切线方程。
\end{proposition}
\begin{proof}
    
\end{proof}

\begin{proposition}
    椭圆:
    \[ \frac{x^2}{a^2} + \frac{y^2}{b^2} = 1 \]
    求椭圆上过$(x_0, y_0)$点的切线。
\end{proposition}
\begin{proof}
    
\end{proof}

% 陈老视频41
$f(x)$在$x_0$处的导数为以下极限:
\[ f'(x_0) = \funclim{\Delta x}{0}{\frac{f(x_0+\Delta x) - f(x_0)}{\Delta x}}\]
称:
\[ f'_+(x_0) = \funclim{\Delta x}{0^+}{\frac{f(x_0+\Delta x) - f(x_0)}{\Delta x}}\]
为$f(x)$在$x_0$的右导数。
称:
\[ f'_-(x_0) = \funclim{\Delta x}{0^-}{\frac{f(x_0+\Delta x) - f(x_0)}{\Delta x}}\]
为$f(x)$在$x_0$的左导数。

因此, $f(x)$在$x_0$可导$\Longleftrightarrow$ $f(x)$在$x_0$的左右导数存在且相等。

以下两个记号不好弄混:$f'_+(x_0)$是$f(x)$在$x_0$的右导数, $f'(x_0^+)$是$f(x)$导数在$x_0$的右极限。

同理, $f'_-(x_0)$是$f(x)$在$x_0$的左导数, $f'(x_0^-)$是$f(x)$导数在$x_0$的左极限。

\begin{proposition}
    $f(x) = \left| x \right|$在$x_0$的左右导数。
\end{proposition}
\begin{proof}
    
\end{proof}

\begin{proposition}
    \begin{equation*}
        f(x) = \left\{ 
            \begin{aligned}
                &x\sin(1/x) &\quad x>0 \\
                &0          &\quad x \le 0
            \end{aligned}
        \right.
    \end{equation*}
    求$f(x)$在$x = 0$的左右导数。
\end{proposition}
\begin{proof}
    
\end{proof}

\begin{proposition}
    \begin{equation*}
        f(x) = \left\{ 
            \begin{aligned}
                & x^2 + b &\quad x > 2 \\
                & ax + 1 &\quad x \le 2
            \end{aligned}
        \right.
    \end{equation*}
    要求确定$a, b$使得$f(x)$在$x_0 = 2$可导。
\end{proposition}
\begin{proof}
    
\end{proof}

$f(x)$在$(a, b)$上每一点可导, 则称$f(x)$在$(a, b)$区间上可导。

$f(x)$在$(a, b)$上每一点可导, 在$x = a$上有右导数, $x = b$又左导数, 则称$f$在闭区间$[a, b]$上可导。

% 陈老视频42
\section{导数四则运算与反函数求导法则}
\begin{proposition}
    求
    \[ y = \sin(x) \]
    的导数。
\end{proposition}
\begin{proof}
    
\end{proof}
同理$y = \cos(x)$, $y'(x) = -\sin(x)$。

\begin{proposition}
    求
    \[ y = \ln(x) \]
    的导数。
\end{proposition}
\begin{proof}
    
\end{proof}

\begin{proposition}
    求
    \[ y = e^x \]
    的导数。
\end{proposition}
\begin{proof}
    
\end{proof}

\begin{proposition}
    求
    \[ y = a^x \]
    的导数。    
\end{proposition}
\begin{proof}
    
\end{proof}

\begin{proposition}
    求
    \[ y = x^\alpha, \alpha \in \mathbb{R} \]
    在定义域$(0, +\infty)$的导数。 
\end{proposition}
\begin{proof}
    
\end{proof}

\begin{theorem}
    若$f$, $g$在同一区间可导, 则$c_1f(x)+c_2g(x)$也在该区间可导, 且有:
    \[ \left(c_1 f(x) + c_2 g(x) \right)' = c_1 f'(x) +c_2 g'(x) \]
\end{theorem}

\begin{theorem}
    若$f$, $g$在同一区间可导, 则$f(x)\dot g(x)$也在该区间可导, 且有:
    \[ \left(f(x) g(x) \right)' = f'(x) g(x) + f(x) g'(x) \]
\end{theorem}
\begin{proof}
    
\end{proof}

\begin{proposition}
    求:
    \[ y = x^3\cos(x) \]
    的导数。
\end{proposition}
\begin{proof}
    
\end{proof}

\begin{proposition}
    求:
    \[ y = \frac{\sin(x)}{x} \]
    的导数。
\end{proposition}
\begin{proof}
    
\end{proof}

\begin{theorem}
    设$g(x)$在某一个区间可导, $g(x) \neq 0$, 则$\frac{1}{g(x)}$也在该区间可导, 且
    \[ \left(\frac{1}{g(x)}\right)' = \frac{-g'(x)}{g^2(x)}\]
\end{theorem}
\begin{proof}
    
\end{proof}

\begin{proposition}
    求:
    \[ y = \sec(x), \left(\sec(x) = \frac{1}{\cos(x)}\right) \]
    的导数。
\end{proposition}
\begin{proof}
    
\end{proof}

\begin{proposition}
    求:
    \[ y = \csc(x), \left( \csc(x) = \frac{1}{\sin(x)} \right) \]
    的导数。
\end{proposition}
\begin{proof}
    
\end{proof}

\begin{lemma}
    $f$, $g$在同一区间可导, $g(x) \neq 0$, 则$\frac{f(x)}{g(x)}$在该区间可导, 且:
    \[ \left( \frac{f(x)}{g(x)} \right)' = \frac{f'(x)g(x)-f(x)g'(x)}{g^2(x)}\]
\end{lemma}
\begin{proof}
    
\end{proof}

\begin{proposition}
    求:
    \[ y = \tan(x) \]
    的导数。
\end{proposition}
\begin{proof}
    
\end{proof}

\begin{proposition}
    求:
    \[ y = \cot(x) \]
    的导数。
\end{proposition}
\begin{proof}
    
\end{proof}

\begin{theorem}[反函数求导定理]
    $f(x)$在$(a, b)${\bf 连续}并且{\bf 严格单调}并且{\bf 可导}, $f'(x) \neq 0$, $\alpha = \min\left( f(a^+), f(b^-)\right)$, $\beta = \max\left( f(a^+), f(b^-)\right)$, 则$f^{-1}(y)$在$(\alpha, \beta)$上可导, 且:
    \[ \left( f^{-1} (y) \right)' = \frac{1}{f'(x)} \]
\end{theorem}
\begin{proof}
    
\end{proof}

\begin{proposition}
    求:
    \[ y = \arctan(x) \]
    的导数。
\end{proposition}    
\begin{proof}
    
\end{proof}

\begin{proposition}
    求:
    \[ y = \arccot(x) \]
    的导数。
\end{proposition}    
\begin{proof}
    
\end{proof}

\begin{proposition}
    求:
    \[ y = \arcsin(x) \]
    的导数。
\end{proposition}    
\begin{proof}
    
\end{proof}

\begin{proposition}
    求:
    \[ y = \arccos(x) \]
    的导数。
\end{proposition}    
\begin{proof}
    
\end{proof}

% 陈老视频43
\begin{proposition}
    考虑:
    \[ \sh(x) = \frac{\e^x - \e^{-x}}{2} \quad \text{和} \quad \ch(x) = \frac{\e^x + \e^{-x}}{2} \]
    的导数。
\end{proposition}
\begin{proof}
    
\end{proof}

\begin{proposition}
    考虑:
    \[ \thx(x) = \frac{\sh(x)}{\ch(x)} \quad \text{和} \quad \cth(x) = \frac{\ch(x)}{\sh(x)} \]
    的导数。
\end{proposition}
\begin{proof}
    
\end{proof}

\begin{proposition}
    考虑:
    \[ \sh^{-1}(x) \quad \text{和} \quad \ch^{-1}(x) \]
    的导数。
\end{proposition}
\begin{proof}
    
\end{proof}

\begin{remark}
    \begin{enumerate}
        \item $\left( \sum_{i = 1}^n c_i f_i(x) \right)' = \sum_{i = 1}^n c_i f'_i(x) $
        \item $\prod_{i = 1}^n f_i(x) = \sum_{j=1}^n\left(f_j'(x)\prod_{i = 1, i \neq j}^n f_i(x) \right)$
    \end{enumerate}
\end{remark}

\begin{proposition}
    求
    \[ y = a_nx^n+a_{n-1}x^{n-1}+\cdots+a_1x+a_0 \]
    的导数。
\end{proposition}
\begin{proof}
    
\end{proof}

\begin{proposition}
    求:
    \[ y = \e^x\left(x^2+3x-1\right)\arcsin(x)\]
\end{proposition}
\begin{proof}
    
\end{proof}

% 陈老视频44
\section{复合函数求导法则及其应用}
\begin{proposition}
    $u = g(x)$在$x_0$可导, $g(x_0) = u_0$, $u = f(u)$在$u = u_0$可导, 则$y = f\left(g(x)\right)$在$x = x_0$可导, 且:
    \[ \left[ f\left(g(x)\right)\right]'_{x=x_0} = f'(u_0)g'(x_0) \]
\end{proposition}
\begin{proof}
    有缺陷证明:

    证明:

\end{proof}
复合函数求导法则又叫链式法则。

\begin{example}
    用复合函数求导法则求:
    \[ y = x^\alpha \]
    的导数。
\end{example}
\begin{proof}
    
\end{proof}

\begin{example}
    用复合函数求导法则求:
    \[ y = \e^{\cos(x)} \]
    的导数。
\end{example}
\begin{proof}
    
\end{proof}

\begin{example}
    用复合函数求导法则求:
    \[ y = \sqrt{1+x^2} \]
    的导数。
\end{example}
\begin{proof}
    
\end{proof}

% 陈老视频45
\begin{proposition}
    求:
    \[ y = \e^{\sqrt{1+\cos(x)}} \]
    的导数。
\end{proposition}
\begin{proof}
    
\end{proof}

幂指函数:
\begin{example}
    求:
    \[ y = f(x) = u(x)^{v(x)} \]
    的导数。
\end{example}
\begin{proof}
    
\end{proof}

\begin{example}
    \[ y = \left(\sin(x)\right)^{\cos(x)}\]
\end{example}
\begin{proof}
    
\end{proof}

\begin{theorem}[一阶微分的形式不变性]
    设$y = f(u)$, 则$y'(u) = f'(u)$, $\D y = f'(u) \D u$, 其中$u$是自变量。

    设$y = f(u), u = g(x)$, 则$y(x) = f(g(x))$, $y'(x) = f'(u)g'(x)$, $y'(x) = f'(g(x))g'(x)$, $\D y = f'(g(x))g'(x) \D x$, 则
    $\D y = f'(g(x)) \D g(x) = f'(u)\D u$, 其中$u$是中间变量。

    无论$u$是自变量还是中间变量,$\D y = f'(u)\D u$
\end{theorem}

\subsection{隐函数的求导与微分}
隐函数:$f(x, y) = 0$。
\begin{example}
    \begin{equation*}
        \frac{x^2}{a^2} + \frac{y^2}{b^2} = 1
    \end{equation*}
    求y关于x的微分。
\end{example}

\begin{example}
    \[ \e^{xy} + x^2y - 1 = 0 \]
    求y关于x的微分。
\end{example}
\begin{proof}
    
\end{proof}

\begin{proposition}
    \[ \sin(y^2) = \cos(\sqrt{x}) \] 
    求y关于x的微分。
\end{proposition}
\begin{proof}
    
\end{proof}

\begin{example}
    \[ \e^{x+y} - xy - e = 0\]
    (0, 1)在曲线上, 求过(0, 1)点的切线方程。
\end{example}
\begin{proof}
    
\end{proof}

\begin{remark}
    \begin{enumerate}
        \item $y \ \frac{1}{g(x)}$也可以看作:
        \begin{equation*}
            \left\{ 
                \begin{aligned}
                    &y = \frac{1}{u} \\
                    &u = g(x)
                \end{aligned}
            \right.
        \end{equation*}
        则$y'(x) = -\frac{1}{g^2(x)}\cdot g'(x)$, 定义证明和复合函数结果一致。
        \item $y = f(x), x = f^{-1}(y)$, 则$f^{-1}((f(x))) = x$, 使用复合函数求导, 则$1 = (f^{-1}(y))'f'(x)$, 即$(f^{-1}(y)) = \frac{1}{f'(x)}$, 用复合函数求导法则可以推导反函数求导。
    \end{enumerate}
\end{remark}

\subsection{函数的参数表示}
函数的参数表示:
\begin{equation*}
    \left\{ 
        \begin{aligned}
            &x = \phi(t) \\
            &y = \psi(t)
        \end{aligned}
    \right.
\end{equation*}
其中$\alpha \le t \le \beta$, $\phi, \psi$可微, $\phi$严格单调, $\phi '(t) \neq 0$。由反函数可导定理t可以表示为$t = \phi^{-1}(x)$,则:
\begin{equation*}
    y = \psi(\phi^{-1}(x))
\end{equation*}
则:
\begin{equation*}
    \frac{\D y}{\D x} = \psi'(t)(\phi'(x))' = \psi'(t)\cdot \frac{1}{\phi(t)} = \frac{\psi'(t)}{\phi'(t)}
\end{equation*}

\begin{example}
    求旋轮线:
    \begin{equation*}
        \left\{ 
            \begin{aligned}
                &x = t - \sin(t) \\
                &y = 1 - \cos(t)
            \end{aligned}
        \right.
    \end{equation*}
    的导数。
\end{example}
\begin{proof}
    
\end{proof}

% 陈老视频46
\begin{example}
    $t = 0$时, 水平速度与垂直向上的速度分别为$v_1$, $v_2$, 问在什么时刻, 速度的方向是水平的?
\end{example}
\begin{proof}
    
\end{proof}

\begin{example}
    \begin{equation*}
        \frac{x^2}{a^2} + \frac{y^2}{b^2} = 1
    \end{equation*}
    分别用三种表示方法的求导方式求导。
\end{example}
\begin{proof}
    
\end{proof}

% 陈老视频47
\section{高阶导数和高阶微分}
\begin{definition}[高阶导数的定义]
    $y=f(x)$, 若$f'(x)$任然可导, 则记它的导函数为:
    \begin{equation*}
        \left[ f'(x) \right]' = f''(x)
    \end{equation*}
    称它为$f(x)$的二阶导数。也可记为$y''(x)$, $\der{}{x}\left(\der{y}{x}\right) = \dern{2}{y}{x}$, $\der{}{x}\left(\der{f}{x}\right) = \dern{2}{f}{x}$。

    若$f''(x)$仍可导, 则它的导数称为$f(x)$的三阶导数, 记为$f'''(x)$, 也可以记为$y'''(x)$, $\der{}{x}\left(\dern{2}{y}{x}\right) = \dern{3}{y}{x}$, $\der{}{x}\left(\dern{2}{f}{x}\right) = \dern{3}{f}{x}$。

    \def\tmp#1{f^{(#1)}(x)}
    从四阶开始记为$\tmp{4}, \tmp{5}, \cdots , \tmp{n}, \cdots$
\end{definition}

\begin{definition}
    \def\tmp#1{f^{(#1)}(x)}
    \def\tmpa#1{y^{(#1)}(x)}
    设$f$的$n-1$阶导数$\tmp{n-1}$仍然可导, 则它的导数记为$\left[ \tmp{n-1}\right]' = \tmp{n}$, 也可记为$\tmpa{n}, \dern{n}{f}{x}, \dern{n}{y}{x}$
\end{definition}

\begin{example}
    求
    \begin{equation*}
        y = \e^x
    \end{equation*}
    的高阶导数。
\end{example}
\begin{proof}
    
\end{proof}

\begin{example}
    求
    \begin{equation*}
        y = a^x
    \end{equation*}
    的高阶导数。
\end{example}
\begin{proof}
    
\end{proof}

\begin{example}
    求
    \begin{equation*}
        y = \sin(x)
    \end{equation*}
    的高阶导数。
\end{example}
\begin{proof}
    
\end{proof}