\chapter{微分中值定理极其应用}
% 陈老视频50(2023.01.11)
\section{微分中值定理}
\begin{definition}
    设$f(x)$的定义区间为$(a, b)$, $x_0 \in (a, b)$, 若$\exists, O(x_0, \rho) \subset (a, b)$, 使得$f(x) \le f(x_0), x \in O(x_0, \rho)$,则称$x_0$是$f$的一个极大值点, $f(x_0)$是一个极大值。
\end{definition}
\begin{remark}
    \begin{enumerate}
        \item 极值是局部概念。
        \item 极小值可以大于极大值。
        \item 极值点可以有无穷多个, 例如: $y = \sin(1/x)$。
        \item 极值概念与连续、可导等概念无关。
    \end{enumerate}
\end{remark}

\begin{lemma}[Fermat引理]\label{theorem:fermat}
    设$x_0$是$f(x)$的一个极值点, 若$f$在$x_0$可导, 则$f'(x_0) = 0$
\end{lemma}
\begin{proof}
    
\end{proof}
\begin{remark}
    导数等于0, 并不一定是极值点, 例如$f(x) = x^3$的$x = 0$点。
\end{remark}

\begin{theorem}[Rolle定理]\label{theorem:rolle}
    $f(x)$在闭区间$[a, b]$连续, 在开区间$(a, b)$可导, $f(a) = f(b)$, 则至少存在一个$\xi \in (a, b)$, 使$f'(\xi) = 0$。
\end{theorem}
\begin{proof}
    
\end{proof}

\begin{example}[(Legendre多项式)]
    若有函数:
    \begin{equation*}
        P_n(x) = \frac{1}{2^n n!}\dern{n}{}{x}(x^2-1)^n
    \end{equation*}
    则它在$(-1, 1)$有n个不同的根。
\end{example}
\begin{proof}
    
\end{proof}

\begin{theorem}[Lagrange中值定理]
    $f(x)$在$[a, b]$连续, 在$(a, b)$可导, 则$\exists \xi \in (a, b)$, 使:
    \begin{equation*}
        f'(\xi) = \frac{f(b)-f(a)}{b-a}
    \end{equation*}
\end{theorem}
\begin{proof}
    
\end{proof}
\begin{remark}
    除了以上形式之外, 还能写成别的形式, 例如
    \begin{enumerate}
        \item $f(b)-f(a)=f'(\xi)(b-a)$
        \item $f(b)-f(a)=f'[a+\theta(b-a)](b-a), \theta \in (0, 1)$
        \item $f(x+\Delta x) - f(x) = f'(x+\theta \Delta x)\Delta x, \theta \in (0, 1)$
        \item $\Delta y = f'(x+\theta \Delta x)\Delta x$
    \end{enumerate}
\end{remark}

\begin{example}
    用Lagrange中值定理讨论函数:

    我们已知$f(x) = c \Rightarrow f'(x) = 0$

    现在证明$f'(x) = 0 \Rightarrow f(x) = c$
\end{example}
\begin{proof}
    
\end{proof}

% 陈老视频51(2023.01.12)
\begin{theorem}[一阶导数与函数的单调性关系]
    $f(x)$在区间$I$定义, 且可导, 则$f(x)$在$I$上单调增加的充分必要条件是: $f'(x) > 0, \forall x \in I$。

    若$\forall x \in I$, $f'(x)>0$, 则$f(x)$在$I$上严格单调增加(充分条件)。
\end{theorem}
\begin{proof}
    \framebox{充分性}:

    \framebox{必要性}:

\end{proof}
\begin{remark}
    若$f(x)$在$I$上连续, 除了有限个点$x_1, x_2, \cdots, x_n$之外, $f'(x)>0$, 则$f'(x)$在I上严格单调增加。
\end{remark}

\section{函数的凸性}
convex(凸), convave(凹), 陈老版本将前者定义为下凸, 后者定义为上凸。

几何上, 下凸: 弦在曲线上方; 上凸: 弦在曲线上方。

\begin{definition}
    $f(x)$在区间$I$上有定义, 若$\forall x_1, x_2 \in I$, $\forall \lambda \in (0, 1)$, 成立$f(\lambda x_1 + (1-\lambda)x_2) \le \lambda f(x_1) + (1-\lambda)f(x_2)$, 则称$f(x)$在区间$I$上是下凸函数。
\end{definition}

\begin{theorem}[二阶导数与凸性的关系]
    设$f(x)$在$I$上二阶可导, 则$f(x)$在$I$下凸的充分必要条件是:$f''(x)\ge 0$, $\forall x \in I$。

    若在$I$上有$f''(x) > 0$, 则$f(x)$在$I$上严格下凸。
\end{theorem}
\begin{proof}
    \framebox{必要性}:

    \framebox{充分性}:
\end{proof}

