\chapter{微分中值定理极其应用}
% 陈老视频50(2023.01.11)
\section{微分中值定理}
\begin{definition}
    设$f(x)$的定义区间为$(a, b)$, $x_0 \in (a, b)$, 若$\exists, O(x_0, \rho) \subset (a, b)$, 使得$f(x) \le f(x_0), x \in O(x_0, \rho)$,则称$x_0$是$f$的一个极大值点, $f(x_0)$是一个极大值。
\end{definition}
\begin{remark}
    \begin{enumerate}
        \item 极值是局部概念。
        \item 极小值可以大于极大值。
        \item 极值点可以有无穷多个, 例如: $y = \sin(1/x)$。
        \item 极值概念与连续、可导等概念无关。
    \end{enumerate}
\end{remark}

\begin{lemma}[Fermat引理]\label{theorem:fermat}
    设$x_0$是$f(x)$的一个极值点, 若$f$在$x_0$可导, 则$f'(x_0) = 0$
\end{lemma}
\begin{proof}
    
\end{proof}
\begin{remark}
    导数等于0, 并不一定是极值点, 例如$f(x) = x^3$的$x = 0$点。
\end{remark}

\begin{theorem}[Rolle定理]\label{theorem:rolle}
    $f(x)$在闭区间$[a, b]$连续, 在开区间$(a, b)$可导, $f(a) = f(b)$, 则至少存在一个$\xi \in (a, b)$, 使$f'(\xi) = 0$。
\end{theorem}
\begin{proof}
    
\end{proof}

\begin{example}[(Legendre多项式)]
    若有函数:
    \begin{equation*}
        P_n(x) = \frac{1}{2^n n!}\dern{n}{}{x}(x^2-1)^n
    \end{equation*}
    则它在$(-1, 1)$有n个不同的根。
\end{example}
\begin{proof}
    
\end{proof}

\begin{theorem}[Lagrange中值定理]
    $f(x)$在$[a, b]$连续, 在$(a, b)$可导, 则$\exists \xi \in (a, b)$, 使:
    \begin{equation*}
        f'(\xi) = \frac{f(b)-f(a)}{b-a}
    \end{equation*}
\end{theorem}
\begin{proof}
    
\end{proof}
\begin{remark}
    除了以上形式之外, 还能写成别的形式, 例如
    \begin{enumerate}
        \item $f(b)-f(a)=f'(\xi)(b-a)$
        \item $f(b)-f(a)=f'[a+\theta(b-a)](b-a), \theta \in (0, 1)$
        \item $f(x+\Delta x) - f(x) = f'(x+\theta \Delta x)\Delta x, \theta \in (0, 1)$
        \item $\Delta y = f'(x+\theta \Delta x)\Delta x$
    \end{enumerate}
\end{remark}

\begin{example}
    用Lagrange中值定理讨论函数:

    我们已知$f(x) = c \Rightarrow f'(x) = 0$

    现在证明$f'(x) = 0 \Rightarrow f(x) = c$
\end{example}
\begin{proof}
    
\end{proof}

% 陈老视频51(2023.01.12)
\begin{theorem}[一阶导数与函数的单调性关系]
    $f(x)$在区间$I$定义, 且可导, 则$f(x)$在$I$上单调增加的充分必要条件是: $f'(x) \ge 0, \forall x \in I$。

    若$\forall x \in I$, $f'(x) > 0$, 则$f(x)$在$I$上严格单调增加(充分条件)。
\end{theorem}
\begin{proof}
    \framebox{充分性}:

    \framebox{必要性}:

\end{proof}
\begin{remark}
    若$f(x)$在$I$上连续, 除了有限个点$x_1, x_2, \cdots, x_n$之外, $f'(x)>0$, 则$f'(x)$在I上严格单调增加。
\end{remark}

\subsection{函数的凸性}
convex(凸), convave(凹), 陈老版本将前者定义为下凸, 后者定义为上凸。

几何上, 下凸: 弦在曲线上方; 上凸: 弦在曲线上方。

\begin{definition}
    $f(x)$在区间$I$上有定义, 若$\forall x_1, x_2 \in I$, $\forall \lambda \in (0, 1)$, 成立$f(\lambda x_1 + (1-\lambda)x_2) \le \lambda f(x_1) + (1-\lambda)f(x_2)$, 则称$f(x)$在区间$I$上是下凸函数。
\end{definition}

\begin{theorem}[二阶导数与凸性的关系]
    设$f(x)$在$I$上二阶可导, 则$f(x)$在$I$下凸的充分必要条件是:$f''(x)\ge 0$, $\forall x \in I$。

    若在$I$上有$f''(x) > 0$, 则$f(x)$在$I$上严格下凸。
\end{theorem}
\begin{proof}
    \framebox{必要性}:

    \framebox{充分性}:
\end{proof}

% 陈老视频52(2023.01.13)
\subsection{拐点}
(拐点会使得作图像样)
\begin{theorem}
    $f(x)$在区间$I$上连续, $(x_0-\delta, x_0+\delta) \subset I$:
    \begin{enumerate}
        \item $f(x)$在$(x_0-\delta, x_0)$与$(x_0, x_0+\delta)$上都二阶可导, $f''(x)$在$(x_0-\delta, x_0)$与$(x_0, x_0+\delta)$上符号相反, 则$(x_0, f(x_0))$是拐点。$f(x)$在$(x_0-\delta, x_0)$与$(x_0, x_0+\delta)$上都二阶可导, $f''(x)$在$(x_0-\delta, x_0)$与$(x_0, x_0+\delta)$上符号相同, 则$(x_0, f(x_0))$不是拐点。
        \item 若$f(x)$在$(x_0-\delta, x_0+\delta)$上二阶可导, $(x_0, f(x_0))$是曲线的拐点, 则$f''(x_0) = 0$。
    \end{enumerate}
\end{theorem}
\begin{proof}
    
\end{proof}

\begin{example}
    求曲线$y = \sqrt[3]{x^2}(x^2-4x)$的拐点。
\end{example}
\begin{proof}
    
\end{proof}

\begin{theorem}[Jensen不等式]
    $f(x)$在区间$I$下凸, 则对于$\forall x_1, x_2, \cdots, x_n \in I$, $\forall \lambda_1, \lambda_2, \cdots, \lambda_n, (\lambda_i > 0, \sum_i \lambda_i = 1)$, 成立:
    \begin{equation*}
        f(\sum_i\lambda_i x_i) \le \sum_i\lambda_i f(x_i)
    \end{equation*}
\end{theorem}
\begin{proof}
    
\end{proof}

\begin{example}
    取$f(x) = \ln(x)$, 证明:
    \begin{equation*}
        \mean{x}{n} > \sqrt[n]{x_1x_2\cdots x_n}
    \end{equation*}
\end{example}
\begin{proof}
    
\end{proof}

\begin{example}
    证明:
    \begin{equation*}
        \left| \arctan(b) - \arctan(a) \right| \le \left| b - a \right|
    \end{equation*}
\end{example}
\begin{proof}
    
\end{proof}

\begin{example}
    证明等式:
    \begin{equation*}
        \arctan\left( \frac{1+x}{1-x} \right) - \arctan(x) = \left\{
            \begin{aligned}
                \frac{\pi}{4} \quad x < 1 \\
                -\frac{3\pi}{4} \quad x > 1
            \end{aligned} \right.
    \end{equation*}
\end{example}
\begin{proof}
    
\end{proof}

\begin{example}
    判断$\e^{\pi}$和$\pi^\e$的大小。
\end{example}
\begin{proof}
    
\end{proof}

% 陈老视频53(2023.01.15)
\begin{example}
    证明: 当$x > 0$时候, 
    \begin{equation*}
        \sin(x) > x - \frac{1}{6}x^3
    \end{equation*}
\end{example}
\begin{proof}
    
\end{proof}

\begin{example}
    证明: 当$a, b > 0$时:
    \begin{equation*}
        a\ln(a) + b\ln(b) \ge (a+b)[\ln(a+b) - \ln2]
    \end{equation*}
\end{example}
\begin{proof}
    
\end{proof}

\begin{example}
    $a, b \ge 0$, $p, q > 0$, 满足$\frac{1}{p} + \frac{1}{q} = 1$, 则:
    \begin{equation*}
        ab \le \frac{1}{p}a^p + \frac{1}{q}b^q
    \end{equation*}    
\end{example}
\begin{proof}

\end{proof}

\begin{theorem}[Cauchy中值定理]\label{theorem:Cauchy-mean-value}
    $f(x), g(x)$在$[a, b]$上连续, 在$(a, b)$上可导, 对$\forall x \in (a, b)$, $g'(x) \neq 0$, 则至少存在$\xi \in (a, b)$, 使得:
    \begin{equation*}
        \frac{f'(\xi)}{g'(\xi)} = \frac{f(b)-f(a)}{g(b) - g(a)}
    \end{equation*}
\end{theorem}
\begin{proof}
    \framebox{证法一}:

    \framebox{证法二}:

\end{proof}

\begin{example}
    设$f(x)$在$[1, +\infty)$连续, 在$(1, +\infty)$可导, $\e^{-x^2}f'(x)$在$(1, +\infty)$上有界,则$x\e^{-x^2}f(x)$也在$(1, +\infty)$上有界。
\end{example}
\begin{proof}
    
\end{proof}

% 陈老视频54(2023.01.16)
\section{L'Hospital法则}
L'Hospital是求待定型的一种重要方法(有的书翻译成洛必达, 有的书翻译成罗比塔)。
\begin{theorem}[L'Hospital法则]~\label{theorem:L'Hospital-law}
    $f(x)$, $g(x)$在$(a, a+d]$上可导, $g'(x) \neq 0$, 若这时有:
    \begin{equation*}
        \funclim{x}{a^+}{f(x)} = \funclim{x}{a^+}{g(x)} = 0
    \end{equation*}
    或者:
    \begin{equation*}
        \funclim{x}{a^+}{g(x)} = \infty(\text{没要求}\funclim{x}{a^+}{f(x)} = \infty)
    \end{equation*}
    且$\funclim{x}{a^+}{\frac{f'(x)}{g'(x)}} = A(\text{或}\infty)$, 则:
    \begin{equation*}
        \funclim{x}{a^+}{\frac{f(x)}{g(x)}} = \funclim{x}{a^+}{\frac{f'(x)}{g'(x)}}
    \end{equation*}
\end{theorem}
\begin{remark}
    \begin{enumerate}
        \item $x \to a^+$也适用于$x \to x_0, x \to x_0^-, x \to \pm \infty, x \to \infty $。
        \item $\funclim{x}{a^+}{\frac{f'(x)}{g'(x)}}$, 也适用于$\frac{f'(x)}{g'(x)} \to \infty, +\infty, -\infty$。
        \item $\funclim{x}{a^+}{\frac{f(x)}{g(x)}}$, 是$\frac{0}{0}$型或者$\frac{*}{\infty}$型。
    \end{enumerate}
\end{remark}
\begin{proof}
    情况一:

    情况二:
\end{proof}

\begin{example}
    求:
    \begin{equation*}
        \funclim{x}{0}{\frac{1-\cos(2x)}{x^2}}
    \end{equation*}
\end{example}
\begin{proof}
    
\end{proof}

\begin{example}
    求:
    \begin{equation*}
        \funclim{x}{\infty}{\frac{\frac{\pi}{2}-\arctan(x)}{\sin{\frac{1}{x}}}}
    \end{equation*}
\end{example}
\begin{proof}
    
\end{proof}

\begin{example}
    求:
    \begin{equation*}
        \funclim{x}{0}{\frac{x-\tan(x)}{x^3}}
    \end{equation*}
\end{example}
\begin{proof}
    
\end{proof}

\begin{example}
    求:
    \begin{equation*}
        \funclim{x}{+\infty}{\frac{x^a}{\e^{bx}}} \quad (a > 0, b > 0)
    \end{equation*}
\end{example}
\begin{proof}
    
\end{proof}

% 陈老视频55(2023.01.17)
\begin{example}
    求:
    \begin{equation*}
        \funclim{x}{0^+}{x\ln(x)}
    \end{equation*}
\end{example}
\begin{proof}
    
\end{proof}

\begin{example}
    求:
    \begin{equation*}
        \funclim{x}{0^+}{\cot(x)- \frac{1}{x}}
    \end{equation*}
\end{example}
\begin{proof}
    
\end{proof}

\begin{example}
    求:
    \begin{equation*}
        \funclim{x}{0^+}{x^x}
    \end{equation*}
\end{example}
\begin{proof}
    
\end{proof}

\begin{example}
    求:
    \begin{equation*}
        \funclim{x}{x^+}{\ln^x(\frac{1}{x})}
    \end{equation*}
\end{example}
\begin{proof}
    
\end{proof}

\begin{example}
    求:
    \begin{equation*}
        \funclim{x}{\frac{\pi}{2}^+}{(\sin(x))^{\tan(x)}}
    \end{equation*}
\end{example}
\begin{proof}
    
\end{proof}

反例(不可用洛必达):
\begin{example}
    求:
    \begin{equation*}
        \funclim{x}{\frac{\pi}{2}}{\frac{1+\sin(x)}{1-\cos(x)}}
    \end{equation*}
\end{example}
\begin{proof}
    
\end{proof}

\begin{example}
    求:
    \begin{equation*}
        \funclim{x}{\infty}{\frac{x+\cos(x)}{x}}
    \end{equation*}
\end{example}
\begin{proof}
    
\end{proof}

% 陈老视频56(2023.01.18)
\section{Taylor多项式与插值多项式}
\subsection{Taylor多项式}
\begin{theorem}[带Peano余项的Taylor公式]
    设$f(x)$在$x_0$有$n$阶导数, 则在$x_0$的领域, 成立:
    \begin{equation*}
        f(x) = f(x_0) + f'(x_0)(x-x_0) + \frac{1}{2!}f''(x_0)(x-x_0)^2+\cdots+\frac{1}{n!}f^{(n)}(x_0)(x-x_0)^n+r_n(x)
    \end{equation*}
    其中:
    \begin{equation*}
        r_n(x) = o((x-x_0)^n) \quad (x \to x_0)
    \end{equation*}
    设:
    \begin{equation*}
        P_n(x) = f(x_0) + f'(x_0)(x-x_0) + \frac{1}{2!}f''(x_0)(x-x_0)^2+\cdots+\frac{1}{n!}f^{(n)}(x_0)(x-x_0)^n
    \end{equation*}
    $P_n(x)$称为$f$在$x=x_0$处的$n$次Taylor多项式。$r_n(x)$称为$f$在$x=x_0$处的Peano余项。
\end{theorem}
\begin{remark}
    $f(x), f'(x), \cdots, f^{(n-1)}(x)$在$x_0$连续。
\end{remark}
\begin{proof}
    
\end{proof}

% 陈老视频57(2023.01.19)
\begin{theorem}[带Lagrange余项的泰勒公式]
    设$f(x)$在$[a, b]$有$n$阶连续导数, 在$(a, b)$上有$n+1$阶导数, 设$x_0 \in [a, b]$为一定点, 则对任意的$x \in [a, b]$, 有:
    \begin{equation*}
        f(x) = f(x_0) + f'(x_0)(x-x_0) + \frac{1}{2!}f''(x_0)(x-x_0)^2+\cdots+\frac{1}{n!}f^{(n)}(x_0)(x-x_0)^n
    \end{equation*}
    其中:
    \begin{equation*}
        r_n(x) = \frac{f^{(n+1)}(\xi)}{(n+1)!}(x-x_0)^{(n+1)} \quad (\xi\text{在}x, x_0\text{之间} )
    \end{equation*}
\end{theorem}
\begin{remark}
    带Lagrange余项的泰勒公式并不要求$x \to x_0$, 它可以描述一定区间的内的情况。
\end{remark}
\begin{proof}
    \framebox{陈老证明(惊为天人)}:
    \wyc{这部分我们暂不写, 待整体的进度到了再写。}

    \framebox{证明二}:
    设两个辅助函数:
    \begin{equation*}
        \begin{split}
            G(x) &= f(x) - \sum_{k=0}^n\frac{1}{k!}f^{(k)}(x_0)(x - x_0)^k \\
            H(x) &= (x - x_0)^{k+1}
        \end{split}
    \end{equation*}
    并且$G(x_0) = 0$, $H(x_0) = 0$。考察$G(x)$和$H(x)$导数:
    \begin{equation*}
        \begin{split}
            G'(x) &= f'(x) - \sum_{k=0}^{n-1}\frac{1}{k!}f^{(k+1)}(x_0)(x-x_0)^k \\ 
            H'(x) &= (n+1)(x - x_0)^{n}
        \end{split}
    \end{equation*}
    并且有$G'(x_0) = 0$, $H'(x_0) = 0$。依次类推, $G^{(i)}(x)$和$H^{(i)}(x)$为:
    \begin{equation*}
        \begin{split}
            G^{(i)}(x) &= f^{(i)}(x) - \sum_{k=0}^{n-i}\frac{1}{k!}f^{(k+i)}(x_0)(x-x_0)^k \\ 
            H^{(i)}(x) &= \frac{(n+1)!}{(n+1-i)!}(x - x_0)^{n+1-i}            
        \end{split}
    \end{equation*}
    并且$G^{(i)}(x_0) = 0$, $H^{(i)}(x) = 0$。
    
    
    现在不妨设$x > x_0$, 则根据Cauchy中值定理(定理~\ref{theorem:Cauchy-mean-value}):
    \begin{equation*}
        \frac{G(x)}{H(x)} = \frac{G(x) - G(x_0)}{H(x) - H(x_0)} = \frac{G'(\xi_1)}{H'(\xi_1)} = \frac{G'(\xi_2)}{H'(\xi_2)} = \cdots = \frac{G'(\xi_n)}{H'(\xi_n)} = \frac{f^{n}(\xi_n) - f^{(n)}(x_0)}{(n+1)!(\xi_n - x_0)} = \frac{f^{(n+1)}(\xi_{n+1})}{(n+1)!}
    \end{equation*}
    因此:
    \begin{equation*}
        r_n(x) = G(x) = \frac{f^{(n+1)}(\xi_{n+1})}{(n+1)!}(x - x_0)^{n+1}
    \end{equation*}
    % 并且有:
    % \begin{equation*}
    %     \begin{split}
    %         G'(x) &= f'(x) - \sum_{k=0}^{n-1}\frac{1}{k!}f^{(k+1)}(x_0)(x-x_0)^k \\ 
    %         H'(x) &= (n+1)(x - x_0)^{n}
    %     \end{split}
    % \end{equation*}
    
\end{proof}

\begin{remark}
    取$n = 0$:
    \begin{equation*}
        f(x) = f(x_0) + f'(\xi)(x - x_0)
    \end{equation*}
    即Lagrange中值定理, 因此带Lagrange余项的Taylor展开是Lagrange中值定理的推广。
\end{remark}

\subsection{插值多项式}
假设有一个多项式:
\begin{equation*}
    P_n(x) = a_0 + a_1x + a_2x^2 + \cdots + a_nx^n
\end{equation*}
要确定多项式的系数, 则需要$n+1$个条件。

设$f(x)$定义于$[a, b]$上有$n$阶导数, 取$x_0, x_1, x_2, \cdots, x_m \in [a, b]$, 要求$P_n(x)$满足:
\begin{equation*}
    P_n^{(j)}(x_i) = f^{(j)}(x_i) \quad (i = 0, 1, 2, \cdots, m; j = 0, 1, 2, \cdots, n_i-1)
\end{equation*}
即, 在$x_i$点, 有$n_i$个条件。那么假设$n+1 = \sum_{i = 0}^m n_i$, 那么我们就能用这些条件确定$n$阶多项式$P_n$。

现在, 我们用$m_j$表示$f^{(j)}(x)$的条件个数, 那么$n+1 = \sum_{j} m_j$。


现在有两个问题:
\begin{enumerate}
    \item 如何找$P_n(x)$?
    \item 如何求余项$r_n(x)$?
\end{enumerate}

% 陈老视频58(2023.01.21)
\begin{theorem}[插值多项式的余项定理]
    $f(x)$在$[a, b]$上有$n$阶连续导数, 在$(a, b)$上有$n+1$阶导数, $x_0, x_1, x_2, \cdots, x_m \in [a, b]$, 设$P_n(x)$是满足插值条件
    \begin{equation*}
        P_n^{(j)}(x_i) = f^{(j)}(x_i) \quad (i = 0, 1, 2, \cdots, m; j = 0, 1, 2, \cdots, n_i-1)
    \end{equation*}
    的$n$次插值多项式, 则:
    \begin{equation*}
        r_n(x) =f(x) - P_n(x) = \frac{f^{(n+1)}(\xi)}{(n+1)!}\prod_{i=0}^{m}(x-x_i)^{n_i}
    \end{equation*}
    其中$\xi \in (x_{min}, x_{max})$, $x_{min} = \min\{x, x_0, x_1, \cdots, x_m\}$, $x_{max} = \max\{x, x_0, x_1, \cdots, x_m\}$
\end{theorem}
\begin{proof}
    
\end{proof}

现在我们已经证明了余项的公式, 接下来我们来讨论如何找插值多项式$P_n(x)$, 但是关于找插值多项式的内容已经超出了数学分析的内容, 因此我们只讨论两种特殊的多项式:
\begin{enumerate}
    \item \framebox{情况一}: $n_0 = n_1 = \cdots = n_m = 1$
    
    在该情况下, 因为$\sum_{i=0}^{m}n_m = n+1$, 因此$n = m$。
    考虑$\omega_{n+1} = \prod_{i=0}^{n}(x-x_i) = \prod_{i=0}^{m}(x-x_i)$, 现在我们定义基函数:
    \begin{equation*}
        q_k(x) = \frac{\prod_{i=0, i\neq k}^{n}(x-x_i)}{\prod_{i=0, i\neq k}^{n}(x_k-x_i)}
    \end{equation*}
    这样定义的$q_k(x)$是一个$n$次多项式, 并且有以下的性质:
    \begin{equation*}
        q_k(x_i) = \left\{
            \begin{aligned}
                1 \quad i = k\\
                0 \quad i \neq k
            \end{aligned}
         \right.
    \end{equation*}
    现在我们定义:
    \begin{equation*}
        P_n(x) = \sum_{k=0}^{n}f(x_k)q_k(x) = f(x_0)q_0(x)+f(x_1)q_1(x) + \cdots + f(x_n)q_n(x)
    \end{equation*}
    \item \framebox{情况二}: 节点仅一个$x_0$
    
    在该情况下, 插值条件为:
    \begin{equation*}
        P_n^{(j)}(x_0) = f^{(j)}(x_0) \quad (j = 0, 1, 2, \cdots , n)
    \end{equation*}
    定义基函数$g_k(x) = \frac{(x-x_0)^k}{k!}$, 那么基函数$q_k(x)$有如下性质:
    \begin{equation*}
        q_k^{(j)}(x_0) = \left\{
            \begin{aligned}
                0 \quad j < k \\
                1 \quad j = k \\
                0 \quad j > k
            \end{aligned}
        \right.
    \end{equation*}
    那么我们就可以用基函数定义:
    \begin{equation*}
        P_n(x) = \sum_{k=0}^{n}f^{(k)}(x_0)q_k(x) = \sum_{k=0}^{n}\frac{f^{(k)}(x_0)}{k!}(x-x_0)^k
    \end{equation*}
    这是$f$在$x_0$的Taylor多项式。
\end{enumerate}
