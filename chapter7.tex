\chapter{定积分}
% 陈老视频76(2023.02.10)
设$f(x)$定义在区间$(a, b)$上, 讨论$f(x)$在$[a, b]$上的定积分, 首先必须要求$f(x)$有界。无界的情况是之后的推广, 现在阶段我们只考虑有界情况。
\begin{definition}[定积分的定义]
    设$f(x)$在$[a, b]$有界, 对$[a, b]$作划分$P$: $a = x_0 < x_1 < x_2 < \cdots < x_n = b $, 任取$\xi_i \in [x_{i-1}, x_i]$, 若极限
    \begin{equation*}
        \funclim{\lambda}{0}{\sum_{i=1}^nf(\xi_i)\Delta x_i} \quad (\lambda = \max\{\Delta x_i\})
    \end{equation*}
    存在, 且极限值与划分$P$及取点$\xi_i$无关。则称$f(x)$在闭区间$[a, b]$上Riemann可积, 简称可积。极限值$I$称为$f(x)$在$[a, b]$上的定积分, 记为
    \begin{equation*}
        \funclim{\lambda}{0}{\sum_{i=1}^nf(\xi_i)\Delta x_i} = I = \int_{a}^{b}f(x) \D x
    \end{equation*}
    
    其中极限中的和式称为Riemann和。
\end{definition}
\begin{remark}
    定义中, $a < b$, 若$a \ge b$, 规定$\int_{a}^{b}f(x)\D x = -\int_{b}^{a}f(x)\D x$。根据以上的定义可知$\int_a^af(x)\D x = 0$。
\end{remark}

\begin{definition}[Riemann可积的``$\epsilon$-$\delta$''语言]
    $\exists I$, $\forall \epsilon > 0$, $\exists \delta > 0$, 使得$\forall P$: $a < x_0 < x_1 < x_2 < \cdots < x_n = b$, $\forall \xi_i \in [x_{i-1}, x_i]$, 只需$\lambda = \max\{\Delta x_i\} < \delta$, 成立
    \begin{equation*}
        \left| \sum_{i=1}^{n}f(\xi_i)\Delta x_i - I \right| < \epsilon
    \end{equation*}
    则$f(x)$在$[a, b]$上可积。
\end{definition}

\begin{example}[Dirichlet函数]
    证明
    \begin{equation*}
        D(x) = 
        \begin{cases}
            1 & x\text{是有理数} \\
            0 & x\text{是无理数}
        \end{cases}
    \end{equation*}
    在$[0, 1]$区间上不可积。
\end{example}
\begin{proof}
    
\end{proof}

% 陈老视频77(2023.02.12)
现在我们要讨论哪些函数是可积的, 哪些函数是不可积的。在讨论函数是可积的还是不可积的之前我们先介绍Darboux和(达布和)。
\begin{definition}
    $f(x)$在$[a, b]$闭区间上有界, 设$M=\sup_{x\in [a, b]} f(x)$, $m=\inf_{x\in [a, b]} f(x)$, 存在划分$P$: $a = x_0 < x_1 < x_2 < \cdots < x_n = b$, 并且设$M_i = \sup_{x\in [x_{i-1}, x_i]} f(x)$, $m_i=\inf_{x\in [x_{i-1}, x_i]} f(x)$, 则称
    \begin{equation*}
        \bar{S}(P) = \sum_{i=0}^n M_i\Delta x_i
    \end{equation*}
    为达布大和, 称
    \begin{equation*}
        \underbar{S}(P) = \sum_{i=0}^n m_i\Delta x_i
    \end{equation*}
    为达布小和。
\end{definition}

\begin{lemma}
    若在原来的划分中增加分点, 则大和不增, 小和不减。
\end{lemma}
\begin{proof}
    
\end{proof}

\begin{lemma}
    对任意的划分$P_1$, $P_2$,
    \begin{equation*}
        m(b-a) \le \underbar{S}(P_2) \le \bar{S}(P_2) \le M(b-a)
    \end{equation*}
\end{lemma}
\begin{proof}
    
\end{proof}
下面我们做一些规定:
\begin{enumerate}
    \item $\bar{S}$表示对一切划分的达布大和构成的集合。
    \item $\underbar{S}$表示对一切划分的达布小和构成的集合。
\end{enumerate}
再令$L = \inf\{\bar{S}(P) | \bar{S}(P) \in \bar{S}\}$, $l = \sup\{\underbar{S}(P) | \underbar{S}(P) \in \underbar{S}\}$, 那么则有$\underbar{S}(P) \le l \le L \le \bar{S}(P_1)$

\begin{theorem}[Darboux定理]
    $f(x)$在$[a, b]$有界, 则有
    \begin{equation*}
        \funclim{\lambda}{0}{\bar{S}(P)} = L \quad \funclim{\lambda}{0}{\underbar{S}(P)} = l
    \end{equation*}
\end{theorem}
\begin{proof}
    
\end{proof}

% 陈老视频78(2023.02.14)

\section{可积的充要条件}
\begin{theorem}[可积的充要条件]
    $f(x)$在$[a, b]$上可积的充分必要条件是:
    \begin{equation*}
        \funclim{\lambda}{0}{\bar{S}(P)} = \funclim{\lambda}{0}{\underbar{S}(P)} \Longleftrightarrow L = l
    \end{equation*}
\end{theorem}
\begin{proof}
    \framebox{必要性}

    \framebox{充分性}
\end{proof}

$M_i-m_i$表示了$f(x)$在$[x_{i-1}, x_i]$上的振幅, 记为$\omega_i$。
\begin{theorem}
    $f(x)$在$[a, b]$上可积的充分必要条件是:
    \begin{equation*}
        \funclim{\lambda}{0}{\sum_{i=0}^n\omega_i\Delta x_i} = 0
    \end{equation*}
\end{theorem}

\begin{example}[Dirichlet函数]
    用以上的定理证明:
    \begin{equation*}
        D(x) = 
        \begin{cases}
            1 & x\text{是有理数} \\
            0 & x\text{是无理数}
        \end{cases}
    \end{equation*}
    在$[0, 1]$区间上不可积。
\end{example}
\begin{proof}
    
\end{proof}

\begin{lemma}
    闭区间上的连续函数一定可积。
\end{lemma}
\begin{proof}
    
\end{proof}

\begin{lemma}
    闭区间上的单调函数一定可积。
\end{lemma}
\begin{proof}
    
\end{proof}

\begin{theorem}
    在$[a, b]$上有界的函数$f(x)$可积的充分必要条件是: $\forall \epsilon > 0$, $\exists P$, 使得$\sum_{i=0}^n\omega_i \Delta x_i < \epsilon$。
\end{theorem}

% 陈老视频79(2023.02.15)
\begin{lemma}
    闭区间上只有有限个不连续点的有界函数一定可积。
\end{lemma}
\begin{proof}
    
\end{proof}

\begin{example}
    Direchlet函数不可积, Riemman函数在$[0, 1]$上可积。
    
    Riemman函数的表达式为:
    \begin{equation*}
        R(x) = 
        \begin{cases}
            \frac{1}{p} &\quad x = \frac{q}{p} \quad (p in N^+, q \in N^+, p, q\text{互质}) \\
            1 &\quad x = 0 \\
            0 &\quad x$是无理数$    
        \end{cases}
    \end{equation*}
\end{example}
\begin{proof}
    
\end{proof}

\begin{example}
    振幅不能任意小的小区间长度之和可以任意小的话, 则$f(x)$可积。
\end{example}
\begin{proof}
    
\end{proof}

% 陈老视频80(2023.02.18)
\section{定积分的基本性质}
\subsection{线性性}
\begin{theorem}[线性性]
    $f(x)$, $g(x)$在$[a, b]$上可积, 则$k_1f(x)+k_2g(x)$也在$[a, b]$上可积, 且
    \begin{equation*}
        \int_{a}^{b}[k_1f(x)+k_2g(x)]\D x = k_1\int_{a}^{b}f(x) \D x + k_2\int_{a}^{b}g(x) \D x
    \end{equation*}
\end{theorem}
\begin{proof}
    
\end{proof}

\begin{lemma}
    $f(x)$在$[a, b]$上可积, $f$与$g$仅在有限个点上取不同的值, 则$g(x)$也在$[a, b]$上可积, 且
    \begin{equation*}
        \int_{a}^{b}g(x)\D x = \int_{a}^{b}f(x)\D x
    \end{equation*} 
\end{lemma}
\begin{proof}
    
\end{proof}

乘积可积性
\begin{theorem}[乘积可积性]
    $f(x)$, $g(x)$在$[a, b]$上可积, 则$f(x)g(x)$在$[a, b]$上可积。
\end{theorem}
\begin{proof}
    
\end{proof}
\begin{remark}
    \begin{equation*}
        \int_{a}^{b}f \cdot g \D x = \int_{a}^{b}f \D x \dot \int_{a}^{b}g \D x
    \end{equation*}
\end{remark}

保序性
\begin{theorem}[保序性]
    $f(x)$, $g(x)$在$[a, b]$上可积, $f(x) \ge g(x)$, 则
    \begin{equation*}
        \int_{a}^{b} f(x) \D x \ge \int_{a}^{b} g(x) \D x
    \end{equation*}
\end{theorem}
\begin{proof}
    
\end{proof}

绝对可积性
\begin{theorem}[绝对可积性]
    $f(x)$在$[a, b]$上可积, 则$|f(x)|$也在$[a, b]$上可积, 且
    \begin{equation*}
        \left|\int_{a}^{b}f(x) \D x \right| \le \int_{a}^{b}|f(x)| \D x
    \end{equation*}
\end{theorem}
\begin{proof}
    
\end{proof}
\begin{remark}
    $f(x)$可积 $\Rightarrow$ $|f(x)|$可积, $|f(x)|$可积 $\nRightarrow$ $f(x)$可积
\end{remark}

区间可加性
\begin{theorem}[区间可加性]
    设$f(x)$在$[a, b]$上可积, $c \in (a, b)$, 则$f(x)$在$[a, c]$与$[c, b]$上可积。反之, 若$f(x)$在$[a, c]$, $[c, b]$上可积, 则$f(x)$在[a, b]上可积, 且
    \begin{equation*}
        \int_{a}^{b}f(x) \D x = \int_{a}^{c}f(x) \D x + \int_{c}^{b}f(x) \D x
    \end{equation*}
\end{theorem}
\begin{proof}
    
\end{proof}

% 陈老视频80(2023.02.18)
\begin{remark}
    $c$点不一定要在$[a, b]$内部, $c$点在$[a, b]$外部也成立。
\end{remark}

积分第一中值定理
\begin{theorem}[积分第一中值定理]
    $f(x)$, $g(x)$在$[a, b]$上可积, $g(x)$在$[a, b]$上不变号, 则$\exists \eta \in [m, M], M = \sup_{x \in [a, b]} f(x), m = \inf_{x \in [a, b]} f(x)$, 使
    \begin{equation*}
        \int_{a}^{b}f(x)g(x)\D x = \eta \int_{a}^{b}g(x)\D x
    \end{equation*}
    若$f(x)$在$[a, b]$上连续, 则$\exists \xi \in [a, b]$, 使
    \begin{equation*}
        \int_a^b f(x)g(x) \D x = f(\xi)\int_a^b g(x) \D x
    \end{equation*}
\end{theorem}
\begin{proof}
    
\end{proof}
\begin{remark}
    考虑$g(x) = 1$, 则
    \begin{equation*}
        \int_a^b f(x) g(x) \D x = \begin{cases}
            \eta(b-a) &\quad f\text{可积} \\
            f(\xi)(b-a) &\quad f\text{连续}
        \end{cases}
    \end{equation*}
    其中$\eta \in [m, M]$, $\xi \in [a, b]$。
\end{remark}

\begin{example}
    $f(x)$在$[a, b]$连续, $f(x) > 0$, 证明
    \begin{equation*}
        \frac{1}{b-a}\int_a^b \ln f(x) \D x \le \ln\left(\frac{1}{b-a}\int_a^b f(x) \D x\right)
    \end{equation*}
\end{example}
\begin{proof}

\end{proof}

\begin{example}[Holder不等式]
    $f(x)$, $g(x)$在$[a, b]$上连续, $p, q > 0$, $\frac{1}{p}+\frac{1}{q} = 1$, 则
    \begin{equation*}
        \int_a^b |f(x)q(x)| \D x = \left(\int_a^b |f(x)|^p \D x\right)^{\frac{1}{p}}\left(\int_a^b |g(x)|^q \D x\right)^{\frac{1}{q}}
    \end{equation*}
\end{example}
\begin{proof}
    
\end{proof}

% 陈老视频81(2023.02.19)
\section{微积分基本定理(Newton-Libniz公式)}
\begin{example}[变速运动]
    物体的速度为$v(t)$, 走过的路程为$S(t)$, 求在时间段$[T_1, T_2]$中物体走过的路程。
\end{example}
\begin{solution}
    取划分$P: T_1 = t_0 < t_1 < t_2 \cdots < t_n = T2, \forall \xi_i \in [t_{i-1}, t_i]$, 则
    \begin{equation*}
        \int_{T_1}^{T_2} = \funclim{\lambda}{0}{\sum_{i=1}^{n}v(\xi_i) \Delta t_i} = S(T_2) - S(T_1)
    \end{equation*}
    其中$S'(t) = v(t)$
\end{solution}

% 陈老视频82(2023.02.20)
\begin{theorem}
    设$f(x)$在$[a, b]$可积, 作函数
    \begin{equation*}
        F(x) = \int_{a}^{x} f(t) \D t \quad x \in [a, b]
    \end{equation*}
    则
    \begin{enumerate}
        \item $F(x)$是$[a, b]$上的连续函数。
        \item 若$f(x)$在$[a, b]$上连续, 则$f(x)$可导, $F'(x) = f(x)$
    \end{enumerate}
\end{theorem}
\begin{proof}
    
\end{proof}

\begin{example}
    令$x > 0$
    \begin{align}
        F(x) &= \int_0^x \sin\sqrt{t} \D t , F'(x) = \sin\sqrt{x} \\
        F(x) &= \int_0^{x^2}\sin\sqrt{t} \D t \\
        F(x) &= \int_{x^2}^1\sin\sqrtsign{t} \D t \\
        F(x) &= \int_{x^2}^x\sin\sqrtsign{t} \D t
    \end{align}
\end{example}
\begin{solution}
    
\end{solution}

\begin{example}
    求:
    \begin{equation*}
        \funclim{x}{0^+}{\frac{\int_{0}^{x^2}\sin\sqrt{t}\D t}{x^3}}
    \end{equation*}
\end{example}

\begin{solution}
    
\end{solution}

\begin{theorem}[微积分基本定理]
    设$f(x)$在$[a, b]$连续, $F(x)$是$f(x)$的一个原函数, 则
    \begin{equation*}
        \int_a^b f(x)\D x = F(b) - F(a)
    \end{equation*}
\end{theorem}
\begin{proof}
    
\end{proof}

\begin{example}
    求:
    \begin{equation*}
        \int_{0}^{1}x^2 \D x
    \end{equation*}
\end{example}
\begin{solution}
    
\end{solution}

\begin{example}
    求:
    \begin{equation*}
        \int_0^\pi \sin x \D x
    \end{equation*}
\end{example}
\begin{solution}
    
\end{solution}

% 陈老视频83(2023.02.21)
\begin{example}
    求:
    \begin{equation*}
        \def\tmp{\frac{1}{n+1}+\frac{1}{n+2}+\cdots+\frac{1}{2n}}
        \funclim{x}{\infty}{\tmp}
    \end{equation*}
\end{example}
\begin{solution}
    
\end{solution}

\section{定积分的分部积分法}
\begin{align}
    &\int_a^b u \D v = uv\Big|_a^b -\int_a^b v \D u \\
    &\int_a^b u(x)v'(x) \D x = u(x)v(x)\Big|_a^b -\int_a^b v(x)u'(x) \D x
\end{align}

\begin{example}
    求
    \begin{equation*}
        y = x\sin x \quad x \in [0, \pi]
    \end{equation*}
    与$x$轴所围区域面积。
\end{example}
\begin{solution}
    
\end{solution}

\begin{definition}
    $\collect{g_n(x)}, n = 0, 1, 2 \cdots$是$[a, b]$上一列函数, 若
    \begin{equation*}
        \int_a^b g_m(x) g_n(x) \D x = \begin{cases}
            0 &\quad m \neq n \\
            \int_a^b g_n^2(x) \D x > 0 &\quad m = n 
        \end{cases}
    \end{equation*}
    则称$\collect{g_n(x)}$是$[a, b]$上的正交函数列。

    若$g_n(x)$是多项式$P_n(x)$, 则称$\collect{P_n(x)}$是$[a, b]$上的正交多项式列。
\end{definition}

\begin{example}[Lagandre多项式]
    对于
    \begin{equation*}
        P_n(x) = \frac{1}{2^n n!}\frac{\D^n}{\D x^n}(x^2-1)^n
    \end{equation*}
    证明
    \begin{equation*}
        \int_{-1}^1 P_m(x)P_n(x) \D x = \begin{cases}
            0 &\quad m \neq n \\
            \frac{2}{2n+1} &\quad m = n
        \end{cases}
    \end{equation*}
\end{example}
\begin{solution}
    
\end{solution}

\begin{example}\label{example-chapter7-1}
    求
    \begin{equation*}
        I_n = \int_0^{\frac{\pi}{2}} \sin^n x \D x
    \end{equation*}
\end{example}
\begin{solution}
    
\end{solution}

\section{定积分换元法}
\begin{theorem}[定积分换元法]
    设$f(x)$在$[a, b]$上连续, $x = \phi(t)$在$[\alpha, \beta]$(或$[\beta, \alpha]$)上有连续导数, $\phi$的值域包含在$[a, b]$中, $\phi(\alpha) = a, \phi(\beta) = b$, 则
    \begin{equation*}
        \int_a^b f(x) \D x = \int_\alpha^\beta f(\phi(t))\phi'(t)\D t
    \end{equation*}
\end{theorem}
\begin{remark}
    \begin{equation*}
        \int_a^b f(x) \D x \text{是否等于} \int_\alpha^\beta f(\phi(t))\phi'(t)\D t
    \end{equation*}
    需要证明。
\end{remark}
\begin{proof}
    
\end{proof}

\begin{example}
    求
    \begin{equation*}
        \int_1^2 \frac{\D x}{x(1+x^4)}
    \end{equation*}
\end{example}
\begin{solution}
    \framebox{解法一}:

    \framebox{解法二}:
\end{solution}
\begin{remark}
    \begin{enumerate}
        \item 对于$\frac{\D x}{x}$, 我们可以取$x^n = t$, 那么此时$n\ln x = \ln t$, $\frac{n}{x}\D x = \frac{\D t}{t}$。
        \item 
    \end{enumerate}
\end{remark}

\begin{example}
    求
    \begin{equation*}
        \int_0^{\frac{\pi}{2}} \sin^3x\cos^4x \D x
    \end{equation*}
\end{example}
\begin{solution}
    
\end{solution}
\begin{remark}
    \begin{enumerate}
        \item $\int_\alpha^\beta \sin^mx\cos^nx \D x$, $m, n$中只要有一个奇数, 就可按上面的方法计算。
        \item 若$m, n$都是偶数, 例如$\int_\alpha^\beta \sin^4x\cos^2x \D x$, 则可以通过三角公式降低次数。
        \begin{equation*}
            \int_\alpha^\beta \sin^4x\cos^2x \D x = \frac{1}{8}\int_\alpha^\beta 2\sin^2x 4\sin^2x\cos^2x \D x = \frac{1}{8} \int_\alpha^\beta (1-\cos2x)sin^22x \D x
        \end{equation*}
        \item 若$\alpha = 0, \beta = \frac{\pi}{2}$, 则可用例\ref{example-chapter7-1}的结论。
    \end{enumerate}
\end{remark}

\begin{example}
    求半径为$r$的圆面积。
\end{example}
\begin{solution}
    
\end{solution}
\begin{remark}
    \begin{enumerate}
        \item $\int_a^bf(x)\D x = \int_\alpha^\beta f(\phi(t))\phi'(t)\D t$, 该式对$\phi(t)$没有单调的要求。而在求不定积分时, $\int f(x)\D x = \int f(\phi(t))\phi'(t)\D t$ 有单调要求, 因为最后还是要换回$x$的函数。
        \item $\phi(t)$的值域在$[a, b]$中, 若$\phi(t)$的值域超出该范围则没定义了, 但是在实际应用中, 如果$\phi(t)$连续, 则超出也可以。
    \end{enumerate}
\end{remark}

% 陈老视频85(2023.02.24)
\begin{example}
    求
    \begin{equation*}
        \int_{\ln2}^1 \frac{\D x}{\sqrt{\e^x-1}}
    \end{equation*}
\end{example}
\begin{solution}
    
\end{solution}

\begin{example}
    有
    \begin{equation*}
        f(x) = \begin{cases}
            \sin\frac{x}{2} &\quad x \ge 0 \\
            x\arctan x &\quad x < 0
        \end{cases}
    \end{equation*}
    求$\int_{0}^{\pi+1}f(x-1)\D x$。                                                            
\end{example}
\begin{solution}
    
\end{solution}

\begin{example}
    求
    \begin{equation*}
        \int_0^1 \frac{\ln(1+x)}{1+x^2} \D x
    \end{equation*}
\end{example}

% 陈老视频85(2023.02.25)
\begin{example}
    求
    \begin{equation*}
        I = \int_{0}^{\frac{\pi}{2}} \frac{\sin^2x}{\sin x+\cos x} \D x
    \end{equation*}
\end{example}
\begin{solution}
    
\end{solution}

\begin{example}
    求
    \begin{equation*}
        \int_0^2 \frac{(x-1)^2}{(x-1)^2+x^2(x-2)^2} \D x
    \end{equation*}
\end{example}
\begin{solution}
    
\end{solution}

\begin{theorem}
    $f(x)$定义与$[-a, a]$上, 可积, 则
    \begin{enumerate}
        \item $f(x)$是奇函数, 则$\int_{-a}^a f(x)\D x = 0$
        \item $f(x)$是偶函数, 则$\int_{-a}^af(x) \D x = 2\int_0^af(x) \D x$
    \end{enumerate}
\end{theorem}
\begin{proof}
    
\end{proof}

\begin{theorem}
    $\collect{1, \sin x, \cos x, \sin 2x, \cos 2x, \cdots, \sin nx, \cos nx, \cdots}$是任意长度为$2\pi$的区间上的正交函数列。
\end{theorem}
\begin{proof}
    
\end{proof}

% 陈老视频86(2023.02.26)
\section{定积分在几何中的应用}
\subsection{平面图形的面积}
$y = f(x), f(x) \ge 0$, 则由$y=f(x), x=a, x=b, y=0$围成的图像的面积为: $A=\int_a^b f(x)\D x$, 若$f(x) < 0$, 则$A = -\int_a^b f(x)\D x$, 若$f(x)$不定号, 则$A=\int_a^c f(x)\D x - int_c^b f(x)\D x = \int_a^b |f(x)| \D x$。

假设有$y = f(x), y = g(x), f(x) \ge g(x)$, 则由$y=f(x), y=g(x), x=a, x=b$围成的图像的面积为: $A = \int_a^b (f(x)-g(x)) \D x$, 若$f(x)$于$g(x)$大小不定, 则$A = \int_a^b|f(x)-g(x)|\D x$。

\begin{example}
    $y = x^2$与$y = \sqrt{x}$所围区域的面积。
\end{example}
\begin{solution}
    
\end{solution}

\begin{example}
    存在等轴双曲线$x^2-y^2 = 1$, 设$(x, y)$是双曲线上的一点, 求$(0, 0)$与$(x, y)$, $(x, -y)$所围图形的面积。
\end{example}
\begin{solution}
    
\end{solution}

\subsubsection{参数表示下求面积}
设曲线
\begin{equation*}
    \begin{cases}
        x = x(t) \\
        y = y(t)
    \end{cases}
    t \in [T_1, T_2]
\end{equation*}
$x(t)$有连续导数, $x'(t) \neq 0$。假设$a = x(T_1), b = x(T_2)$, 则面积为:
\begin{equation*}
    A = \int_a^b f(x) \D x = \int_{T_1}^{T_2} f(x(t))x'(t) \D t = \int_{T_1}^{T_2} y(t) x'(t) \D t
\end{equation*}

\begin{example}
    求椭圆
    \begin{equation*}
        \frac{x^2}{a^2} + \frac{y^2}{b^2} = 1
    \end{equation*}
    的面积。
\end{example}
\begin{solution}
    
\end{solution}

\begin{example}
    求旋轮线
    \begin{equation*}
        \begin{cases}
            x = a(t - \sin t) \\
            y = a(1 - \cos t)
        \end{cases}
        t \in [0, 2\pi]
    \end{equation*}
    与$y = 0$所围的面积。
\end{example}
\begin{solution}
    
\end{solution}

\subsubsection{极坐标表示下求面积}
设$r = r(\theta), \theta \in [\alpha, \beta]$, 现在我们要求$\collect{r(\theta) | \alpha \le \theta \le \beta, 0 \le r \le r(\theta)}$的面积。