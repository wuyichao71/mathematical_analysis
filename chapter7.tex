\chapter{定积分}
% 陈老视频75(2023.02.10)
设$f(x)$定义在区间$(a, b)$上, 讨论$f(x)$在$[a, b]$上的定积分, 首先必须要求$f(x)$有界。无界的情况是之后的推广, 现在阶段我们只考虑有界情况。
\begin{definition}[定积分的定义]
    设$f(x)$在$[a, b]$有界, 对$[a, b]$作划分$P$: $a = x_0 < x_1 < x_2 < \cdots < x_n = b $, 任取$\xi_i \in [x_{i-1}, x_i]$, 若极限
    \begin{equation*}
        \funclim{\lambda}{0}{\sum_{i=1}^nf(\xi_i)\Delta x_i} \quad (\lambda = \max\{\Delta x_i\})
    \end{equation*}
    存在, 且极限值与划分$P$及取点$\xi_i$无关。则称$f(x)$在闭区间$[a, b]$上Riemann可积, 简称可积。极限值$I$称为$f(x)$在$[a, b]$上的定积分, 记为
    \begin{equation*}
        \funclim{\lambda}{0}{\sum_{i=1}^nf(\xi_i)\Delta x_i} = I = \int_{a}^{b}f(x) \D x
    \end{equation*}
    
    其中极限中的和式称为Riemann和。
\end{definition}
\begin{remark}
    定义中, $a < b$, 若$a \ge b$, 规定$\int_{a}^{b}f(x)\D x = -\int_{b}^{a}f(x)\D x$。根据以上的定义可知$\int_a^af(x)\D x = 0$。
\end{remark}

\begin{definition}[Riemann可积的``$\epsilon$-$\delta$''语言]
    $\exists I$, $\forall \epsilon > 0$, $\exists \delta > 0$, 使得$\forall P$: $a < x_0 < x_1 < x_2 < \cdots < x_n = b$, $\forall \xi_i \in [x_{i-1}, x_i]$, 只需$\lambda = \max\{\Delta x_i\} < \delta$, 成立
    \begin{equation*}
        \left| \sum_{i=1}^{n}f(\xi_i)\Delta x_i - I \right| < \epsilon
    \end{equation*}
    则$f(x)$在$[a, b]$上可积。
\end{definition}

\begin{example}[Dirichlet函数]
    证明
    \begin{equation*}
        D(x) = 
        \begin{cases}
            1 & x\text{是有理数} \\
            0 & x\text{是无理数}
        \end{cases}
    \end{equation*}
    在$[0, 1]$区间上不可积。
\end{example}
\begin{proof}
    
\end{proof}

现在我们要讨论哪些函数是可积的, 哪些函数是不可积的。在讨论函数是可积的还是不可积的之前我们先介绍Darboux和(达布和)。
\begin{definition}
    $f(x)$在$[a, b]$闭区间上有界, 设$M=\sup_{x\in [a, b]} f(x)$, $m=\inf_{x\in [a, b]} f(x)$, 存在划分$P$: $a = x_0 < x_1 < x_2 < \cdots < x_n = b$, 并且设$M_i = \sup_{x\in [x_{i-1}, x_i]} f(x)$, $m_i=\inf_{x\in [x_{i-1}, x_i]} f(x)$, 则称
    \begin{equation*}
        \bar{S}(P) = \sum_{i=0}^n M_i\Delta x_i
    \end{equation*}
    为达布大和, 称
    \begin{equation*}
        \underbar{S}(P) = \sum_{i=0}^n m_i\Delta x_i
    \end{equation*}
    为达布小和。
\end{definition}

\begin{lemma}
    若在原来的划分中增加分点, 则大和不增, 小和不减。
\end{lemma}
\begin{proof}
    
\end{proof}

\begin{lemma}
    对任意的划分$P_1$, $P_2$,
    \begin{equation*}
        m(b-a) \le \underbar{S}(P_2) \le \bar{S}(P_2) \le M(b-a)
    \end{equation*}
\end{lemma}
\begin{proof}
    
\end{proof}

