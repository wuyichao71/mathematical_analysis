\chapter{定积分}
% 陈老视频76(2023.02.10)
设$f(x)$定义在区间$(a, b)$上, 讨论$f(x)$在$[a, b]$上的定积分, 首先必须要求$f(x)$有界。无界的情况是之后的推广, 现在阶段我们只考虑有界情况。
\begin{definition}[定积分的定义]
    设$f(x)$在$[a, b]$有界, 对$[a, b]$作划分$P$: $a = x_0 < x_1 < x_2 < \cdots < x_n = b $, 任取$\xi_i \in [x_{i-1}, x_i]$, 若极限
    \begin{equation*}
        \funclim{\lambda}{0}{\sum_{i=1}^nf(\xi_i)\Delta x_i} \quad (\lambda = \max\{\Delta x_i\})
    \end{equation*}
    存在, 且极限值与划分$P$及取点$\xi_i$无关。则称$f(x)$在闭区间$[a, b]$上Riemann可积, 简称可积。极限值$I$称为$f(x)$在$[a, b]$上的定积分, 记为
    \begin{equation*}
        \funclim{\lambda}{0}{\sum_{i=1}^nf(\xi_i)\Delta x_i} = I = \int_{a}^{b}f(x) \D x
    \end{equation*}
    
    其中极限中的和式称为Riemann和。
\end{definition}
\begin{remark}
    定义中, $a < b$, 若$a \ge b$, 规定$\int_{a}^{b}f(x)\D x = -\int_{b}^{a}f(x)\D x$。根据以上的定义可知$\int_a^af(x)\D x = 0$。
\end{remark}

\begin{definition}[Riemann可积的``$\epsilon$-$\delta$''语言]
    $\exists I$, $\forall \epsilon > 0$, $\exists \delta > 0$, 使得$\forall P$: $a < x_0 < x_1 < x_2 < \cdots < x_n = b$, $\forall \xi_i \in [x_{i-1}, x_i]$, 只需$\lambda = \max\{\Delta x_i\} < \delta$, 成立
    \begin{equation*}
        \left| \sum_{i=1}^{n}f(\xi_i)\Delta x_i - I \right| < \epsilon
    \end{equation*}
    则$f(x)$在$[a, b]$上可积。
\end{definition}

\begin{example}[Dirichlet函数]
    证明
    \begin{equation*}
        D(x) = 
        \begin{cases}
            1 & x\text{是有理数} \\
            0 & x\text{是无理数}
        \end{cases}
    \end{equation*}
    在$[0, 1]$区间上不可积。
\end{example}
\begin{proof}
    
\end{proof}

% 陈老视频77(2023.02.12)
现在我们要讨论哪些函数是可积的, 哪些函数是不可积的。在讨论函数是可积的还是不可积的之前我们先介绍Darboux和(达布和)。
\begin{definition}
    $f(x)$在$[a, b]$闭区间上有界, 设$M=\sup_{x\in [a, b]} f(x)$, $m=\inf_{x\in [a, b]} f(x)$, 存在划分$P$: $a = x_0 < x_1 < x_2 < \cdots < x_n = b$, 并且设$M_i = \sup_{x\in [x_{i-1}, x_i]} f(x)$, $m_i=\inf_{x\in [x_{i-1}, x_i]} f(x)$, 则称
    \begin{equation*}
        \bar{S}(P) = \sum_{i=0}^n M_i\Delta x_i
    \end{equation*}
    为达布大和, 称
    \begin{equation*}
        \underbar{S}(P) = \sum_{i=0}^n m_i\Delta x_i
    \end{equation*}
    为达布小和。
\end{definition}

\begin{lemma}
    若在原来的划分中增加分点, 则大和不增, 小和不减。
\end{lemma}
\begin{proof}
    
\end{proof}

\begin{lemma}
    对任意的划分$P_1$, $P_2$,
    \begin{equation*}
        m(b-a) \le \underbar{S}(P_2) \le \bar{S}(P_2) \le M(b-a)
    \end{equation*}
\end{lemma}
\begin{proof}
    
\end{proof}
下面我们做一些规定:
\begin{enumerate}
    \item $\bar{S}$表示对一切划分的达布大和构成的集合。
    \item $\underbar{S}$表示对一切划分的达布小和构成的集合。
\end{enumerate}
再令$L = \inf\{\bar{S}(P) | \bar{S}(P) \in \bar{S}\}$, $l = \sup\{\underbar{S}(P) | \underbar{S}(P) \in \underbar{S}\}$, 那么则有$\underbar{S}(P) \le l \le L \le \bar{S}(P_1)$

\begin{theorem}[Darboux定理]
    $f(x)$在$[a, b]$有界, 则有
    \begin{equation*}
        \funclim{\lambda}{0}{\bar{S}(P)} = L \quad \funclim{\lambda}{0}{\underbar{S}(P)} = l
    \end{equation*}
\end{theorem}
\begin{proof}
    
\end{proof}

% 陈老视频78(2023.02.14)

\section{可积的充要条件}
\begin{theorem}[可积的充要条件]
    $f(x)$在$[a, b]$上可积的充分必要条件是:
    \begin{equation*}
        \funclim{\lambda}{0}{\bar{S}(P)} = \funclim{\lambda}{0}{\underbar{S}(P)} \Longleftrightarrow L = l
    \end{equation*}
\end{theorem}
\begin{proof}
    \framebox{必要性}

    \framebox{充分性}
\end{proof}

$M_i-m_i$表示了$f(x)$在$[x_{i-1}, x_i]$上的振幅, 记为$\omega_i$。
\begin{theorem}
    $f(x)$在$[a, b]$上可积的充分必要条件是:
    \begin{equation*}
        \funclim{\lambda}{0}{\sum_{i=0}^n\omega_i\Delta x_i} = 0
    \end{equation*}
\end{theorem}

\begin{example}[Dirichlet函数]
    用以上的定理证明:
    \begin{equation*}
        D(x) = 
        \begin{cases}
            1 & x\text{是有理数} \\
            0 & x\text{是无理数}
        \end{cases}
    \end{equation*}
    在$[0, 1]$区间上不可积。
\end{example}
\begin{proof}
    
\end{proof}

\begin{lemma}
    闭区间上的连续函数一定可积。
\end{lemma}
\begin{proof}
    
\end{proof}

\begin{lemma}
    闭区间上的单调函数一定可积。
\end{lemma}
\begin{proof}
    
\end{proof}

\begin{theorem}
    在$[a, b]$上有界的函数$f(x)$可积的充分必要条件是: $\forall \epsilon > 0$, $\exists P$, 使得$\sum_{i=0}^n\omega_i \Delta x_i < \epsilon$。
\end{theorem}

% 陈老视频79(2023.02.15)
\begin{lemma}
    闭区间上只有有限个不连续点的有界函数一定可积。
\end{lemma}
\begin{proof}
    
\end{proof}

\begin{example}
    Direchlet函数不可积, Riemman函数在$[0, 1]$上可积。
    
    Riemman函数的表达式为:
    \begin{equation*}
        R(x) = 
        \begin{cases}
            \frac{1}{p} &\quad x = \frac{q}{p} \quad (p in N^+, q \in N^+, p, q\text{互质}) \\
            1 &\quad x = 0 \\
            0 &\quad x$是无理数$    
        \end{cases}
    \end{equation*}
\end{example}
\begin{proof}
    
\end{proof}

\begin{example}
    振幅不能任意小的小区间长度之和可以任意小的话, 则$f(x)$可积。
\end{example}
\begin{proof}
    
\end{proof}

% 陈老视频80(2023.02.18)
\section{定积分的基本性质}
\subsection{线性性}
\begin{theorem}[线性性]
    $f(x)$, $g(x)$在$[a, b]$上可积, 则$k_1f(x)+k_2g(x)$也在$[a, b]$上可积, 且
    \begin{equation*}
        \int_{a}^{b}[k_1f(x)+k_2g(x)]\D x = k_1\int_{a}^{b}f(x) \D x + k_2\int_{a}^{b}g(x) \D x
    \end{equation*}
\end{theorem}
\begin{proof}
    
\end{proof}

\begin{lemma}
    $f(x)$在$[a, b]$上可积, $f$与$g$仅在有限个点上取不同的值, 则$g(x)$也在$[a, b]$上可积, 且
    \begin{equation*}
        \int_{a}^{b}g(x)\D x = \int_{a}^{b}f(x)\D x
    \end{equation*} 
\end{lemma}
\begin{proof}
    
\end{proof}

乘积可积性
\begin{theorem}[乘积可积性]
    $f(x)$, $g(x)$在$[a, b]$上可积, 则$f(x)g(x)$在$[a, b]$上可积。
\end{theorem}
\begin{proof}
    
\end{proof}
\begin{remark}
    \begin{equation*}
        \int_{a}^{b}f \cdot g \D x = \int_{a}^{b}f \D x \dot \int_{a}^{b}g \D x
    \end{equation*}
\end{remark}

保序性
\begin{theorem}[保序性]
    $f(x)$, $g(x)$在$[a, b]$上可积, $f(x) \ge g(x)$, 则
    \begin{equation*}
        \int_{a}^{b} f(x) \D x \ge \int_{a}^{b} g(x) \D x
    \end{equation*}
\end{theorem}
\begin{proof}
    
\end{proof}

绝对可积性
\begin{theorem}[绝对可积性]
    $f(x)$在$[a, b]$上可积, 则$|f(x)|$也在$[a, b]$上可积, 且
    \begin{equation*}
        \left|\int_{a}^{b}f(x) \D x \right| \le \int_{a}^{b}|f(x)| \D x
    \end{equation*}
\end{theorem}
\begin{proof}
    
\end{proof}
\begin{remark}
    $f(x)$可积 $\Rightarrow$ $|f(x)|$可积, $|f(x)|$可积 $\nRightarrow$ $f(x)$可积
\end{remark}

区间可加性
\begin{theorem}[区间可加性]
    设$f(x)$在$[a, b]$上可积, $c \in (a, b)$, 则$f(x)$在$[a, c]$与$[c, b]$上可积。反之, 若$f(x)$在$[a, c]$, $[c, b]$上可积, 则$f(x)$在[a, b]上可积, 且
    \begin{equation*}
        \int_{a}^{b}f(x) \D x = \int_{a}^{c}f(x) \D x + \int_{c}^{b}f(x) \D x
    \end{equation*}
\end{theorem}
\begin{proof}
    
\end{proof}

% 陈老视频80(2023.02.18)
\begin{remark}
    $c$点不一定要在$[a, b]$内部, $c$点在$[a, b]$外部也成立。
\end{remark}

积分第一中值定理
\begin{theorem}[积分第一中值定理]
    $f(x)$, $g(x)$在$[a, b]$上可积, $g(x)$在$[a, b]$上不变号, 则$\exists \eta \in [m, M], M = \sup_{x \in [a, b]} f(x), m = \inf_{x \in [a, b]} f(x)$, 使
    \begin{equation*}
        \int_{a}^{b}f(x)g(x)\D x = \eta \int_{a}^{b}g(x)\D x
    \end{equation*}
    若$f(x)$在$[a, b]$上连续, 则$\exists \xi \in [a, b]$, 使
    \begin{equation*}
        \int_a^b f(x)g(x) \D x = f(\xi)\int_a^b g(x) \D x
    \end{equation*}
\end{theorem}
\begin{proof}
    
\end{proof}
\begin{remark}
    考虑$g(x) = 1$, 则
    \begin{equation*}
        \int_a^b f(x) g(x) \D x = \begin{cases}
            \eta(b-a) &\quad f\text{可积} \\
            f(\xi)(b-a) &\quad f\text{连续}
        \end{cases}
    \end{equation*}
    其中$\eta \in [m, M]$, $\xi \in [a, b]$。
\end{remark}

\begin{example}
    $f(x)$在$[a, b]$连续, $f(x) > 0$, 证明
    \begin{equation*}
        \frac{1}{b-a}\int_a^b \ln f(x) \D x \le \ln\left(\frac{1}{b-a}\int_a^b f(x) \D x\right)
    \end{equation*}
\end{example}
\begin{proof}

\end{proof}

\begin{example}[Holder不等式]
    $f(x)$, $g(x)$在$[a, b]$上连续, $p, q > 0$, $\frac{1}{p}+\frac{1}{q} = 1$, 则
    \begin{equation*}
        \int_a^b |f(x)q(x)| \D x = \left(\int_a^b |f(x)|^p \D x\right)^{\frac{1}{p}}\left(\int_a^b |g(x)|^q \D x\right)^{\frac{1}{q}}
    \end{equation*}
\end{example}
\begin{proof}
    
\end{proof}