\documentclass[lang=cn]{elegantbook}
% \usepackage[utf8]{inputenc}

\newcommand\wyc[1]{\bf\textcolor{cyan}{#1}}
\newcommand{\listsum}[3]{#1_0 #2^#3 + #1_1 #2^{#3 - 1} + \cdots + #1_{#3 - 1} #2 + #1_k}
\newcommand{\mean}[2]{\frac{#1_1 + #1_2 + \cdots + #1_#2}{#2}}
\newcommand{\mylim}[2]{\lim_{#1 \to \infty } #2}
\newcommand{\mylinelim}[2]{\lim\limits_{#1 \to \infty} #2}
\newcommand{\collect}[1]{\left\{ #1\right\}}
\newcommand{\funclim}[3]{\lim_{#1 \to #2 } #3}
\newcommand{\arccot}{\mathrm{arccot}}
\newcommand{\e}{\mathrm{e}}
\newcommand{\D}{\mathrm{d}}
\newcommand{\ch}{\mathrm{ch}}
\newcommand{\sh}{\mathrm{sh}}
% \th \eth is defined
\newcommand{\thx}{\mathrm{th}}
\newcommand{\cth}{\mathrm{cth}}

\newcommand{\der}[2]{\frac{\D #1}{\D #2}}
\newcommand{\dern}[3]{\frac{\D^#1 #2}{\D #3^#1}}


\title{数学分析}
\author{hapo}
\date{\today}

\setcounter{tocdepth}{3}
\logo{nju-emblem.pdf}
\cover{cover.jpg}

\begin{document}

\maketitle
\frontmatter

\tableofcontents

\mainmatter
记录本书的原因是因为我觉得自己的数学分析实在太烂了。因此我决定使用费曼学习法。这是一个链式反应,在我学习随机过程的时候,我发现自己的概率论太差了,而当我学习概率论的时候,我又发现我的测度论太差了,而我学习测度论的时候,我最终发现,我差的是数学分析。而这个最初始的问题是我并没有系统地学过数学分析。而在学习数学分析的过程中,我发现我对好多概念并不明晰,这使得学习的进度缓慢,并且经常容易在概念上卡壳,而卡壳结束后又很快忘记。因此,为了能够更好地记住,我决定使用费曼学习法。在此,我记录下我学习数学分析的过程。
\chapter{集合与映射}
% 陈老视频1
\begin{definition}[集合]
{\bf 集合(集)}是具有某种特定性质的具体的或抽象的对象汇集的总体,其中的对象称为集合的元素。
\end{definition}
集合通常记为A, B, C, X, Y

元素通常记为s, t, a, b, x, y

x是集合S的元素,记为$x\in S$
\chapter{实数的完备性}
\begin{introduction}
  \item 有理数的定义~\ref{def:rational-number}

\end{introduction}
在本章中,我们将介绍实数的完备性。在介绍实数之前,我们需要将实数定义或者说引入。而对于实数的定义是有很多的,而我在此介绍无穷十进制小数表示和戴德金(Dedekind)分割。这部分我将结合陈纪修教授的教材和Ayumu的讲义。

在引入实数之前,我们需要先介绍有理数,而在介绍有理数之前,我们需要规定一些常用集合的记号。而介绍常用集合,则需要先介绍集合(set),在此我们并不介绍集合,我们暂时默认我们已经知道了集合的概念,将来我们会对该部分内容进行扩充。现在让我们来列举一些常用的集合。
\begin{definition}[常用集合表示]
$\mathbb{N} = \{ 1, 2, 3, \cdots, n, \cdots\}$

$\mathbb{Z} = \{ 0, \pm 1, \pm 2, \pm 3, \cdots, \pm n, \cdots \}$

$\mathbb{Z}^+ = \{ n | n \in \mathbb{Z}, n > 0 \}$
\end{definition}
在定义了以上的集合表示之后,我们就能以上的符号来表示有理数:
\begin{definition}[有理数] \label{def:rational-number} 
若一个数x可以表示成$\frac{q}{p}$的形式,其中$q \in \mathbb{Z}$,$p \in \mathbb{Z}^+$, 则称x为{\bf 有理数}(rational number)。而由有理数组成的集合称为{\bf 有理数集},有理数集常用$\mathbb{Q}$表示,其可以表示为:
\[ \mathbb{Q} = \{ x | x = \frac{q}{p}, q \in \mathbb{Z}, p \in \mathbb{Z}^+ \} \]
在这里,我们可以看到p只需要属于$\mathbb{Z}^+$, 这是因为若x为负的, 我们总可以规定负号出现在分子上。
\end{definition}
有理数对加减乘除都封闭,并且我们在有理数上定义了大小关系,即有理数也是良序的。但是它并不能满足研究所需。例如,存在长度无法用有理数表示。并且有理数之间存在``空隙'',即有理数不连续。而在我们之后的研究中,我们往往需要研究连续性,因此我们需要对有理数进行扩充。在这之前,我们先来证明确实有一些数无法用有理数表示。


\begin{proposition}
$\sqrt{2}$不是有理数。
\end{proposition}
\begin{proof}
我们用反证法。假设$\sqrt{2}$为有理数,那么存在$q \in \mathbb{Z}, p \in \mathbb{Z}^+$,使得$\sqrt{2} = \frac{q}{p}$。
\end{proof}

而且有的函数在实数域和有理数域上的表现完全不一样,比如狄利克雷(dirichlet)函数。

% 陈老视频8
\begin{theorem}[确界存在定理---实数系连续性定理]
    非空有上界的数集必有上确界,非空有下界的数集必有下确界。
\end{theorem}

% 陈老视频9
\section{数列极限}
\begin{definition}[数列极限的定义]
    对于数列$\{ a_n \}$, 存在一个实常数a, 对$\forall \epsilon > 0$, $\exists N \in \mathbb{Z}^+$,使得当$n > N$时,$|a_n-a| < \epsilon$成立,则称$\{ a_n \}${\bf 收敛}(convengent)于a或者$\{ a_n \}$的{\bf 极限}(limit)为a, 记作
    \[ \lim_{n \to \infty} a_n = a \text{\quad 或者 \quad}  a_n \to a(n \to +\infty)\]
    若不存在实数a, 满足上述性质, 则称数列$\{ a_n \}${\bf 发散}(divergent)。
\end{definition}
一个数列收敛与否,收敛的话,收敛于哪个数,这与数列的前有限项无关。

\begin{definition}[无穷小量的定义]
    以零为极限的变量称为{\bf 无穷小量}。
\end{definition}

\begin{proposition}
    \[ \mylim{n}{q^n}= 1(q < 1) \]
\end{proposition}
    \begin{proof}
        
    \end{proof}

% 陈老视频10
\begin{proposition}
\[ \lim_{n \to \infty } \sqrt[n]{a}= 1 \]
\end{proposition}
\begin{proof}
    
\end{proof}

\begin{proposition}
\[ \lim_{n \to \infty } \sqrt[n]{n}= 1 \]
\end{proposition}
\begin{proof}
    
\end{proof}


\begin{proposition}
    设$a_n > 0$,$\lim\limits_{n \to \infty} a_n = a$,证明
    \[ \lim_{n \to \infty}\frac{a_1+a_2+\cdots+a_n}{n} = a \]
\end{proposition}
\begin{proof}
    
\end{proof}

\begin{theorem}[数列极限的唯一性]
    若
    \[ \mylim{n}{x_n} = a, \mylim{n}{x_n} = b \]
    则
    \[  a = b \]
\end{theorem}
\begin{proof}
    
\end{proof}


\begin{definition}
    \begin{enumerate}
        \item 对于数列$\collect{x_n}$, 若$\exists M \in \mathbb{R}$, $\forall n \in \mathbb{N}^+$, 成立$x_n \le M$, 则称M是$\collect{x_n}$的一个上界, 或称$\collect{x_n}$有上界。
        \item 对于数列$\collect{x_n}$, 若$\exists m \in \mathbb{R}$, $\forall n \in \mathbb{N}^+$, 成立$x_n \ge m$, 则称m是$\collect{x_n}$的一个下界, 或称$\collect{x_n}$有下界。
    \end{enumerate}
    $\collect{x_n}$既有上界又有下界, 则称$\collect{x_n}$有界。

    $\collect{x_n}$有界的另一个定义:$\exists X \in \mathbb{R}^+$, $\forall x \in \mathbb{N}^+$, 成立$\left| x_n \right| \le X$
\end{definition}

\begin{theorem}[数列极限的有界性]
    若$\{ x_n \}$的极限存在,则$\{ x_n \}$有界。
\end{theorem}
\begin{proof}
    
\end{proof}
% 陈老视频11
\begin{theorem}[数列极限的保序性]
    存在两个数列$\{ x_n \}$和$\{ y_n \}$,并且
    \[ \mylim{n}{x_n} = a, \mylim{n}{y_n} = b, \text{\quad 且 \quad}a < b \]则$\exists N \in \mathbb{Z}^+$,当$n > N$时,$x_n < y_n$。
\end{theorem}
\begin{proof}
    
\end{proof}

\begin{lemma}[数列极限的保序性逆命题]
    对于$\mylim{n}{x_n} = a,\mylim{n}{x_n} = b$, 若$\exists N \in \mathbb{N}^+$, $\forall n > N$, $x_n \le y_n$, 则$a \le b$。 
\end{lemma}
\begin{proof}
    
\end{proof}

\begin{lemma}
    \begin{enumerate}
        \item 若$\mylim{n}{y_n} = b > 0$, 则$\exists N \in \mathbb{N}^+$, $\forall n > N$, $y_n > \frac{b}{2} > 0 $。
        \item 若$\mylim{n}{y_n} = b < 0$, 则$\exists N \in \mathbb{N}^+$, $\forall n > N$, $y_n < \frac{b}{2} < 0 $。
    \end{enumerate}
\end{lemma}
\begin{proof}
    利用数列极限的保序性证明。当$b > 0$时,取$\collect{x_n = \frac{b}{2}}$, 则$\mylim{n}{x_n} = \frac{b}{2} < b$, 由数列极限的保序性可知,$\exists N \in \mathbb{N}^+$, $\forall n > N$, 有$y_n > x_n = \frac{b}{2}$, 证毕。

    同理可证明$b < 0$的情况。
\end{proof}
以上推论可以合起来写为:
\begin{lemma}
    若$\mylim{n}{y_n} = b \neq 0$, 则$\exists N \in \mathbb{N}^+$, $\forall n > N$, $\left| y_n \right|> \frac{\left| b \right|}{2} > 0 $。
\end{lemma}
\begin{proof}
    除了上一个推论那样分开来证明, 还可以如下证明:
    $\mylim{n}{y_n} = b$, 即$\forall \epsilon > 0$, $\exists N \in \mathbb{N}^+$, $\forall n > N$, $\left| y_n - b\right| < \epsilon$。

    由三角不等式得:
    \[ \left| y_n - b\right| > \left| \left| y_n \right| - \left| b \right| \right| \]
    即:
    \[ \left| \left| y_n \right| - \left| b \right| \right| < \left| y_n - b\right| < \epsilon \]
    因此$\mylim{n}{\left| y_n \right|} = \left| b \right|$。之后的证明同上。
\end{proof}

\begin{theorem}[数列极限的夹逼性定理]
    对于数列$\collect{x_n}$, $\collect{y_n}$, $\collect{z_n}$, 若$\exists N \in \mathbb{N}^+$, $\forall n > N$, 成立$x_n \le y_n \le z_n$, 且$\mylim{n}{x_n} = \mylim{n}{z_n} = a$, 则$\mylim{n}{y_n} = a$
\end{theorem}
\begin{proof}
    $\forall \epsilon$, $\exists N_1 \in \mathbb{N}^+$, $\forall n > N_1$, $\left| x_n - a\right| < \epsilon$, 即$x_n -a > -\epsilon$。

    同理, $\forall \epsilon$, $\exists N_2 \in \mathbb{N}^+$, $\forall n > N_2$, $\left| y_n - a\right| < \epsilon$, 即$z_n -a < \epsilon$。

    取$N = \max\left\{ N_1, N_2 \right\}$, 则:
    \[ -\epsilon < x_n - a < y_n - a < z_n - a < \epsilon \]
    即:
    \[ -\epsilon < y_n - a <\epsilon \]
    即:
    \[ \left| y_n - a \right| < \epsilon \]
    总结起来即为:
    $\forall \epsilon$, $\exists N \in \mathbb{N}^+$, $\forall n > N$, $\left| y_n - a\right| < \epsilon$, 即$\mylim{n}{y_n} = a$, 证毕。
\end{proof}
\begin{proposition}
    求:
    \[ \mylim{n}{\sqrt{n+1} - \sqrt{n}}\]
\end{proposition}
% 陈老视频12
\begin{theorem}[数列极限的四则运算]
    若$\lim\limits_{n \to \infty} x_n = a$, $\lim\limits_{n \to \infty} y_n = b$,则
    \begin{enumerate}
        \item $\lim\limits_{n \to \infty}(\alpha x_n + \beta y_n) = \alpha a + \beta b$
        \item $\lim\limits_{n \to \infty}(x_n y_n) = ab$
        \item $\lim\limits_{n \to \infty}(\frac{x_n}{y_n}) = \frac{a}{b}(b \neq 0)$
    \end{enumerate}
\end{theorem}
\begin{proof}
    
\end{proof}

\begin{proposition}
    \[\text{求} \lim\limits_{n \to \infty} \frac{5^{n+1}-(-2)^n}{3 \cdot 5^n + 2\cdot 3^n} \]
\end{proposition}
\begin{proof}
    
\end{proof}

\begin{proposition}
    \[\text{当} a > 0 \text{时,} \lim\limits_{n \to \infty} \sqrt[n]{a} = 1 \]
\end{proposition}
\begin{proof}
    
\end{proof}

\begin{proposition}
    \[\text{求}\lim\limits_{n \to \infty} n\left( \sqrt{n^2+1} - \sqrt{n^2 - 1}\right) \]
\end{proposition}
\begin{proof}
    
\end{proof}

\begin{proposition}
    \[\text{求}\lim\limits_{n \to \infty} \left( \frac{1}{\sqrt{n^2+1}} + \frac{1}{\sqrt{n^2 + 2}} + \cdots \frac{1}{\sqrt{n^2+n}} \right) \]
\end{proposition}
\begin{proof}
    
\end{proof}
有限个

\begin{proposition}
    设$a_n > 0$,$\lim\limits_{n \to \infty} a_n = a$,证明
    \[ \lim_{n \to \infty}\sqrt[n]{a_1a_2\cdots a_n} = a \]
\end{proposition}
\begin{proof}
    
\end{proof}

\begin{proposition}
    若$\{ x_n \}$是无穷小量,$\{ y_n \}$有界($| y_n | < 0$),则$\{ x_n y_n \}$也是无穷小量。
\end{proposition}
\begin{proof}

\end{proof}

\section{无穷大量}
% 陈老视频13
\begin{definition}[无穷大量]
对于一个数列$\{ x_n \}$,若对于任意给定的$G > 0$,可以找到正整数N,使得当$n > N$时,$| x_n |> G$,则称数列$\{ x_n \}$是无穷大量,记为
\[ \lim_{n \to \infty } x_n = \infty \]
若对于该数列$\{ x_n \}$, $| x_n |> G$可以恒表示为$x_n > G$,则称数列$\{ x_n \}$是正无穷大量,记为$+\infty$。

若对于该数列$\{ x_n \}$,$| x_n |> G$可以恒表示为$x_n < -G$,则称数列$\{ x_n \}$是负无穷大量,记为$-\infty$。
\end{definition}

\begin{proposition}
    设$| q | > 1$,证明$\{ q^n \}$是无穷大量。 
\end{proposition}
\begin{proof}
    
\end{proof}

\begin{proposition}
    证明$\{ \frac{n^2-1}{n+5} \}$是无穷大量。 
\end{proposition}
\begin{proof}
    
\end{proof}

\begin{lemma}
    若$x_n \neq 0$,则$\{ x_n \}$是无穷大量的充要条件是$\{ \frac{1}{x_n} \}$是无穷小量。    
\end{lemma}
\begin{proof}

\end{proof}

% 陈老视频14
\begin{lemma}
    设$\{ x_n \}$是无穷大量,$\{ y_n \}$满足$\exists N_0 \in \mathbb{Z}^+$, $\forall n > N_0$,有$| y_n | \ge \delta > 0$,则$\{ x_n y_n\}$也是无穷大量。
\end{lemma}
\begin{proof}
\end{proof}

\begin{lemma}
    设$\{ x_n \}$是无穷大量,$\{ y_n \}$极限存在,且$\lim\limits_{n \to \infty}y_n = b \ne 0$,则$\{ x_n y_n\}$与$\left\{ \frac{x_n}{y_n} \right\}$都是无穷大量。
\end{lemma}
\begin{proof}
\end{proof}

\begin{proposition}
    $\{ \frac{n}{\sin(n)}\}$和$\{ n \cdot \arctan(n)\}$是无穷大量。
\end{proposition}
\begin{proof}
\end{proof}

\begin{proposition}
    讨论极限
    \[ \lim_{n \to \infty} \frac{\listsum{a}{n}{k}}{\listsum{b}{n}{l}}\]
    其中$a_0, b_0 \ne 0$, $k, l \in \mathbb{Z}^+$
\end{proposition}
\begin{proof}
    从分子上提出$n^k$,分母上提出$n^l$次
\end{proof}
\subsection{无穷大的运算}
若$\mylim{n}{x_n} = +\infty$,则记$\collect{x_n}$为``+$\infty$''。

若$\mylim{n}{y_n} = +\infty$,则记$\collect{y_n}$为``+$\infty$''。

若$\mylim{n}{z_n} = \infty$,则记$\collect{z_n}$为``+$\infty$''。

若$\mylim{n}{w_n} = 0$,则记$\collect{z_n}$为``0''。
\begin{theorem}
    \begin{enumerate}
        \item $(+\infty) + (+\infty) = +\infty$
        \item $(+\infty) - (-\infty) = +\infty$
        \item $(+\infty) \pm (\text{有界量}) = +\infty$
        \item $(+\infty) \cdot (+\infty) = +\infty$
        \item $(+\infty) \cdot (-\infty) = -\infty$
    \end{enumerate}
\end{theorem}

\begin{definition}
    \begin{enumerate}
        \item $(+\infty) - (+\infty) = ?$
        \item $(+\infty) + (-\infty) = ?$
        \item $0 \cdot \infty = ?$
        \item $\frac{0}{0} = ?$
        \item $\frac{\infty}{\infty} = ?$
        \item $\cdots$
    \end{enumerate}
    上述情况称为``待定型''
\end{definition}
\begin{definition}
    若数列$\collect{x_n}$,有$x_n \le x_{n + 1}, \forall n \in \mathbb{N}^+$,则称数列$\collect{x_n}$单调增加,记为$\collect{x_n} \uparrow$。若有$x_n < x_{n + 1}$,则称数列$\collect{x_n}$严格单调增加,记为$\collect{x_n} \text{严格} \uparrow$。
    
    若数列$\collect{x_n}$,有$x_n \ge x_{n + 1}, \forall n \in \mathbb{N}^+$,则称数列$\collect{x_n}$单调减少,记为$\collect{x_n} \downarrow$。若有$x_n > x_{n + 1}$,则称数列$\collect{x_n}$严格单调减少,记为$\collect{x_n} \text{严格} \downarrow$。
\end{definition}

\begin{theorem}[Stolz定理]
    假设$\collect{y_n}$严格单调增加数列,且$\mylinelim{n}{y_n} = + \infty$。若
    \[ \mylim{n}{ \frac{x_n - x_{n-1}}{y_n - y_{n-1}} } = a(a\text{为有限数}, +\infty, -\infty) \]
    则
    \[ \mylim{n}{\frac{x_n}{y_n}} = a\]
    
\end{theorem}
\begin{proof}
\end{proof}
\begin{proposition}
    用Stolz定理证明, 若$\mylim{n}{a_n} = a$,则
    \[ \mylim{n}{\mean{a}{n}} = a \]
\end{proposition}
\begin{proof}
\end{proof}

\begin{proposition}
    \[ \text{求} \mylim{n}{\frac{1^k+2^k+\cdots+n^k}{n^{k+1}}} \]
\end{proposition}
\begin{proof}

\end{proof}

\begin{proposition}
    \[ \text{设}\mylim{n}{a_n} = a\text{,求}\mylim{n}{\frac{a_1+2a_2+3a_3+\cdots+na_n}{n^2}} \]    
\end{proposition}
\begin{proof}

\end{proof}

% 陈老视频15
\section{收敛准则}
收敛数列一定有界,但是收敛数列不一定有界。
\begin{enumerate}
    \item 那么有界数列加什么条件收敛?
    \item 有界数列不加条件的情况下,可以得到什么弱一些的结论?
\end{enumerate}
\begin{theorem}
    单调有界数列必定收敛。
\end{theorem}
\begin{proof}
不妨设$\collect{x_n}$单调增加,有上界。
\end{proof}
定理意义:从定义证明时,我们需要知道极限a,相当于验证极限为a,而当极限未知时,则无法证明。而定理则从数列本身的性质出发,不需要知道极限是多少。

\begin{proposition}
    设$x_1 > 0$, $x_{n+1} = 1 + \frac{x_n}{1+x_n}, n = 1,2, 3, \cdots$,证明$\collect{x_n}$收敛,并求极限。
\end{proposition}
\begin{proof}

\end{proof}

\begin{proposition}
    设$ 0 < x_1 < 1$, $x_{n+1} = x_n(1 - x_n), n = 1,2, 3, \cdots$,证明$\collect{x_n}$收敛,并求极限。
\end{proposition}
\begin{proof}

\end{proof}

无穷小量的趋近速度。
\begin{proposition}
    \[ \text{对于上题的}\collect{x_n},\text{求极限,}\mylim{n}{nx_n}\]
\end{proposition}
\begin{proof}

\end{proof}

% 陈老视频16
\begin{proposition}
    $ x_1 = \sqrt{2}, x_{n+1} = \sqrt{3+2x_n}, n = 1, 2, 3, \cdots $,证明$\collect{x_n}$收敛,并求极限。
\end{proposition}
\begin{proof}

\end{proof}

兔子
\begin{proposition}[Fibonacci数列]
    $\collect{a——n}$为Fibonacci数列,令$b_n = \frac{a_{n+1}}{a_n}$,讨论$\collect{b_n}$数列。
    
\end{proposition}
\begin{proof}

\end{proof}

% 陈老视频17
接下来我们来研究$\pi$和$e$

关于$\pi$:
\begin{proposition}
    证明$\collect{L_n = n\sin \left ( \frac{180^\circ}{n} \right )}$收敛。求圆的面积公式。
\end{proposition}
\begin{proof}

\end{proof}

关于$e$:
\begin{proposition}
    考虑两个数列:
    \[ \collect{x_n = \left(1 + \frac{1}{n}\right)^n} \quad \text{和} \quad \collect{y_n = \left(1 + \frac{1}{n}\right)^{n+1}} \]
    证明这两个数列极限存在且相等。
\end{proposition}
\begin{proof}

\end{proof}

定义$\ln = \log_e$为自然对数,$e$自然对数的底数。

\begin{proposition}
    令$a_n = 1 + \frac{1}{2^p} + \frac{1}{3^p} + \cdots + \frac{1}{n^p}, (p > 0)$,证明$\collect{a_n}$当$p>1$时收敛,当$p \le 1$时发散。    
\end{proposition}
\begin{proof}

$p = 1$时,$a_n = 1 + \frac{1}{2} +\frac{1}{3} + \cdots + \frac{1}{n}$为调和级数,它是正无穷大量,我们想知道它趋近无限的速度。
\end{proof}

% 陈老视频18
\begin{proposition}
证明
    \[ b_n = \left( 1 +\frac{1}{2} + \frac{1}{3} + \cdot + \frac{1}{n} \right) - \ln n \]
    收敛。
\end{proposition}
\begin{proof}
    极限记为$\gamma$,称为欧拉常熟。$\gamma >= 0.577215$
\end{proof}

\begin{proposition}
证明
\[ \mylim{n}{\frac{1}{n+1} + \frac{1}{n+2} + \cdots + \frac{1}{2n}} = \ln 2\]
\end{proposition}
\begin{proof}
除了夹逼准则,还能用上一个数列相减计算。
\end{proof}

\begin{proposition}
\[ d_n = 1- \frac{1}{2} + \frac{1}{3} - \frac{1}{4} + \cdots +(-1)^{n+}\frac{1}{n} \]
\end{proposition}
\begin{proof}

\end{proof}
以上是与$e$相关的数列

\begin{definition}[闭区间套]
    有一列闭区间$\collect{\left[a_n, b_n \right]}$,满足:
    \begin{enumerate}
        \item $\left[ a_{n+1}, b_{n+1}\right] \subset \left[ a_n, b_n\right], n = 1, 2, 3, \cdots $
        \item $b_n - a_n \to 0(n \to \infty)$
    \end{enumerate}
    则称这样的一列闭区间是一个闭区间套。
\end{definition}
\begin{theorem}[闭区间套定理]
    假如$\left[ a_n, b_n\right]$是一个闭区间套,则存在唯一的实数$\xi$,它属于一切闭区间$\left[ a_n, b_n\right]$。且$\mylim{n}{a_n} = \mylim{n}{b_n} = \xi$。
\end{theorem}
\begin{proof}
\end{proof}

\begin{theorem}
    实数集不可列。
\end{theorem}
\begin{proof}
    反证法
\end{proof}

% 陈老视频19
子列
\begin{definition}
    存在一个数列$\collect{x_n}$,取一列严格单调增加的正整数$n_1 < n_2 < n_3 < \cdots < n_k < \cdots$,则$x_{n_1}, x_{n_2}, \cdots, x_{n_k}, \cdot$称为$\collect{x_n}$的一个子列,记为$\collect{x_{n_k}}$, $k$代表子列中的第$k$项,又恰好是$\collect{x_n}$中的第$n_k$项。
\end{definition}
其中$n_k \ge k, \forall k$, $n_j > n_k, \forall j > k$。
\begin{theorem}
    设$\collect{x_n}$收敛于a, 则它的任何一个子列也收敛于a。即$\mylim{n}{x_n} = a$,证明$\mylim{k}{a_{n_k}} = a$
\end{theorem}
\begin{proof}

\end{proof}
可以用于证明数列不收敛。
\begin{proposition}
    若$\collect{x_n}$存在两个子列收敛于不同的极限,则$\collect{x_n}$发散。
\end{proposition}
\begin{proof}

\end{proof}

\begin{theorem}[Bolzano-Weierstrass定理]
    有界数列必有收敛子列。
\end{theorem}
\begin{proof}
    
\end{proof}

\begin{theorem}
    假设$\collect{x_n}$是无界数列,则存在子列$\collect{x_{n_k}}$,它是无穷大量。
\end{theorem}
\begin{proof}
    
\end{proof}

% 陈老视频20
Cauchy收敛原理

\begin{definition}
    $\collect{x_n}$满足:
    \[\forall \xi > 0, \exists N, \forall n, m > N, \left|x_n - x_m\right| < \xi \]
    则称$\collect{x_n}$为基本数列。
\end{definition}
也可以是$\forall m > n > N$
\begin{proof}
\end{proof}

\begin{proposition}
    判断
    \[ x_n = 1 + \frac{1}{2^2} + \frac{1}{3^2} + \cdots + \frac{1}{n^2} \]
    是否为基本数列。
\end{proposition}
\begin{proof}

\end{proof}

\begin{proposition}
    判断
    \[ x_n = 1 + \frac{1}{2} + \frac{1}{3} + \cdots + \frac{1}{n} \]
    是否为基本数列。
\end{proposition}
\begin{proof}

\end{proof}

\begin{theorem}[Cauchy收敛原理]
    $\collect{x_n}$收敛的充分必要条件是$\collect{x_n}$是基本数列。
\end{theorem}
\begin{proof}
必要性:

充分性:
\end{proof}

\begin{proposition}
    $\collect{x_n}$满足压缩性条件,即
    \[ \left| x_{n+1} -x_n \right| \le k\left| x_n - x_{n-1}\right|, 0 < k < 1, \forall n = 2, 3, \cdots \] 
    则$\collect{x_n}$是收敛的。
\end{proposition}
\begin{proof}

\end{proof}

% 陈老视频21
实数系的基本定理
\begin{enumerate}
    \item 确界存在定理(实数系的连续性定理)
    \item 单调有界数列收敛定理
    \item 闭区间套定理
    \item Bolzano-Weierstrass定理
    \item Cauchy收敛原理(实数系完备性)
\end{enumerate}
以上都是在实数系中考虑, 实数系上的基本数列必然是收敛数列, 因此Cauchy收敛原理也被称为实数系的完备性定理。实数系有完备性, 有理数不具备完备性。例如$\collect{\left( 1 + \frac{1}{n}\right)^n}$是有理数列, 但是它的极限是无理数。以上五个定理等价。我们需要证明这五个定理等价。从Cauchy收敛原理推出闭区间套定理, 再从闭区间套定理推出确界存在定理。
\begin{theorem}
    实数系的完备性等价于实数系的连续性。
\end{theorem}
\begin{proof}
    (1)Cauchy收敛原理$\Rightarrow$闭区间套定理。

    (2)闭区间套定理$\Rightarrow$确界存在定理。
\end{proof}

% 陈老视频22
\chapter{函数极限与连续函数}
函数极限

\[ \funclim{x}{0}{\frac{\sin x}{x}} = 1 \]
\begin{definition}
    $y = f(x)$在$O(x_0, \rho) \backslash \collect{x_0}$上有定义, 如果存在一个数A, 使得对任意给定的$\epsilon > 0$, 可以找到$\delta > 0$, 当$0 < \left| x - x_0 \right| < \delta$时, 成立$\left| f(x) - A\right| < \epsilon$, 则称A是$f(x)$在$x_0$点的极限,记为$\lim_{x \to x_0} f(x) = A$或者$f(x) \to A (x \to x_0)$。如果不存在满足上述性质的A, 则称$f(x)$在$x_0$点极限不存在。
\end{definition}
$O(x_0, \rho) \backslash \collect{x_0}$称为去心邻域。

% 陈老视频23
\begin{proposition}
    证明:
    \[ \funclim{x}{0}{e^x} = 1\]
\end{proposition}
\begin{proof}

\end{proof}

\begin{proposition}
    证明:
    \[ \funclim{x}{2}{x^2} = 4 \]
\end{proposition}
\begin{proof}

\end{proof}

\begin{proposition}
    证明:
    \[ \funclim{x}{1}{\frac{x(x-1)}{x^2-1}} = \frac{1}{2} \]
\end{proposition}
\begin{proof}
    
\end{proof}

函数极限的性质
\begin{theorem}[函数极限的唯一性]
    设A, B都是$f(x)$在$x_0$的极限, 则$A=B$。 
\end{theorem}
\begin{proof}
    证明类似证明数列极限的唯一性
\end{proof}

\begin{theorem}[函数极限的局部保序性]
    若$\funclim{x}{x_0}{f(x)} = A, \funclim{x}{x_0}{g(x)} = B, A > B$, 则$\exists \delta > 0$, 当x有$0 < \left| x - x_0 \right| < \delta $时, $f(x) > g(x)$  
\end{theorem}
\begin{proof}
    证明类似证明数列极限的保序性
\end{proof}

\begin{lemma}
    $\funclim{x}{x_0}{f(x)} = A \neq 0$, 则$\exists \delta > 0$, $\forall x ( 0 < \left| x - x_0\right| < \delta)$, $\left| f(x) \right| > \frac{\left| A \right|}{2}$。
\end{lemma}
\begin{proof}
    请使用局部保序性证明
\end{proof}

\begin{lemma}
    假设$\funclim{x}{x_0}{f(x)} = A, \funclim{x}{x_0}{g(x)} = B$, 若$\exists \delta > 0, \forall x (x < \left| x - x_0\right| < \delta)$, 有$f(x) \ge g(x)$, 则$A \ge B$
\end{lemma}
\begin{proof}

\end{proof}

\end{document}
